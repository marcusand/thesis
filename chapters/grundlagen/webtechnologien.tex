\section{Webtechnologien}
\label{section:webtechnologien}

Mit der Erfindung des World Wide Web durch Tim Berners-Lee und Robert Cailliau und damit auch der Erfindung 
von HTML \cite{www} entstand auch der Begriff der Webtechnologien.\\
Der Begriff umfasst also all die Technologien, die beteiligt sind um eine Webseite von einem Server über ein 
Netz an einen Client zu übertragen und anzuzeigen.\\

Im Rahmen dieser Arbeit soll der Begriff sich weiter nur auf die Softwaretechnologien beschränken. Während 
sich serverseitig allerhand Programmiersprachen und Technologien finden, sind seit jeher die Softwaretechnologien
auf der Clientseite im Kern auf HTML, CSS und JavaScript beschränkt. Dabei markiert JavaScript als neueste dieser
drei Technologien \cite{jspress} den Beginn der Rich-Client Applikationen und bietet die Möglichkeit im Browser 
vollständige, interaktive Applikationen statt nur statische Seiten anzuzeigen. Und spätestens mit der Erfindung 
von Node.js ist JavaScript keine rein clientseitige Progrmmiersprache mehr. 

% nodejs, npm und enormer Vorritt von JavaScript schon lange keine rein clientseitige Programmiersprache mehr
% sondern multi plattform und mutli use (?) Sprache
% Chromium Embedded Framework

% 