\section{Webtechnologien}
\label{sec:webtechnologien}

Mit der Erfindung des World Wide Web durch Tim Berners-Lee und Robert Cailliau und damit auch der Erfindung 
von HTML \cite{www}, entstand der Begriff der Webtechnologien.\\
Der Begriff umfasst all die Technologien, die beteiligt sind um eine Webseite und Daten von einem Server über ein 
Netz an einen Client zu übertragen und anzuzeigen.\\

Im Rahmen dieser Arbeit soll der Begriff sich weiter nur auf die Softwaretechnologien beschränken. Dabei lassen
sich die Softwaretechnologien klassischerweise in clientseitige und serverseitige Technologien einteilen.\\
Während sich serverseitig allerhand Programmiersprachen und Technologien finden, sind seit jeher die Softwaretechnologien
auf der Clientseite im Kern auf HTML, CSS und JavaScript beschränkt. Dabei markiert JavaScript als neueste dieser
drei Technologien \cite{jspress} den Beginn der Rich-Client Applikationen und bietet die Möglichkeit im Browser 
vollständige, interaktive Applikationen statt nur statische Seiten anzuzeigen. Seitdem hat JavaScript eine rasante 
Entwicklung gemacht und ist heute die populärste Programmiersprache der Welt \cite{npmstat}. Getrieben durch 
die Popularität und die Menge der Entwickler entstand so ein riesiges Ökosystem an Bibliotheken, Frameworks
und Technologien rund um JavaScript \cite{npmstat}. Seitdem kann JavaScript auch nicht mehr als reine Webtechnologie
betrachtet werden.
Denn spätestens mit der Entwicklung und Quellcode-Veröffentlichung der JavaScript-Engine V8 \cite{v8} beschränkt 
sich die Nutzung von JavaScript nicht mehr nur auf den Browser. V8 ist ein in C++ implementierter JavaScript-Interpreter, 
welcher standalone oder in einem C++ Programm eingebunden genutzt werden kann. Auf der V8-Engine beruht beispielsweise 
die asynchrone JavaScript Laufzeitumgebung Node.js, mit der in JavaScript implementierte Programme direkt ausgeführt 
werden können \cite{node}. Genutzt wird Node.js so meist für Serverapplikationen oder serverseitige Microservices.
Auf Node.js und Chromium wiederum basiert das Framework Electron \cite{electron}. Mit diesem können Desktop-Anwendungen
entwickelt werden. Dabei werden Browsertechnologien, sowie serverseitiger JavaScript-Code gleichermaßen
genutzt um vollständige Anwendungen zu implementieren.\\
Die als Webtechnologie bezeichnete Sprache JavaScript hat also mittlerweile einen universellen Einsatz gefunden 
und ist nicht mehr nur eine Technologie für den Browser.\\

So zeigt sich, dass der Begriff Webtechnologien nicht mehr nur in der ursprünglichen Definition gesehen werden kann,
sondern weiter fasst. Vielmehr sind Webtechnologien heute ein Toolkit, welche in vielen Situationen eingesetzt werden
können. Um die Webtechnologien weiter einzuschränken, soll daher der Begriff der \emph{Browsertechnologien} eingeführt werden. 
Er umfasst nur die clientseitigen Technologien, also HTML, CSS und clientseitigen JavaScript-Code 
für die Anzeige in einer Browser-Umgebung.

% nodejs, npm und enormer Vorritt von JavaScript schon lange keine rein clientseitige Programmiersprache mehr
% sondern multi plattform und mutli use (?) Sprache
% Chromium Embedded Framework