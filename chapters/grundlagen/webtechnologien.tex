\section{Webtechnologien}
\label{sec:webtechnologien}

Mit der Erfindung des World Wide Web durch Tim Berners-Lee und Robert Cailliau und damit auch der Erfindung 
von HTML \cite{www} entstand auch der Begriff der Webtechnologien.\\
Der Begriff umfasst also all die Technologien, die beteiligt sind um eine Webseite von einem Server über ein 
Netz an einen Client zu übertragen und anzuzeigen.\\

Im Rahmen dieser Arbeit soll der Begriff sich weiter nur auf die Softwaretechnologien beschränken. Dabei lassen
sich die Softwaretechnologien in clientseitige und serverseitige Technologien einteilen.\\
Während sich serverseitig allerhand Programmiersprachen und Technologien finden, sind seit jeher die Softwaretechnologien
auf der Clientseite im Kern auf HTML, CSS und JavaScript beschränkt. Dabei markiert JavaScript als neueste dieser
drei Technologien \cite{jspress} den Beginn der Rich-Client Applikationen und bietet die Möglichkeit im Browser 
vollständige, interaktive Applikationen statt nur statische Seiten anzuzeigen.\\
Spätestens mit der Entwicklung und Quellcode-Veröffentlichung der JavaScript-Engine V8 \cite{v8} beschränkt sich die Nutzung
von JavaScript auch nicht mehr nur auf den Browser. V8 ist ein in C++ geschriebener JavaScript-Interpreter,\todo{check}
welcher Standalone oder eingebunden in einem C++ Programm genutzt werden kann.\\
Auf der V8-Engine beruht beispielsweise die asynchrone JavaScript Laufzeitumgebung Node.js, mit der in 
JavaScript geschriebene Programm direkt ausgeführt werden können \cite{node}. Genutzt wird Node.js so meist für Serverapplikationen
oder serverseitige Microservices.
Auf Node.js und Chromium wiederum basiert das Framework Electron \cite{electron}. Mit diesem können Desktop-Anwendungen
cross-plattform entwickelt werden.

So zeigt sich, dass der Begriff \emph{Webtechnologien} nicht mehr nur in der ursprünglichen Definition gesehen werden kann,
sondern viel weiter fasst. Vielmehr sind Webtechnologien heute ein Toolkit, welches in fast jeder Situation eingesetzt werden
kann. 

\todo{Überarbeiten und besser ausführen}

% nodejs, npm und enormer Vorritt von JavaScript schon lange keine rein clientseitige Programmiersprache mehr
% sondern multi plattform und mutli use (?) Sprache
% Chromium Embedded Framework
% 