\section{Kiosksysteme}
\label{sec:kiosk}

Wenn man in der Informationstechnologie von Kiosksystemen spricht,
sind damit meist zugängliche Computersysteme gemeint, die im öffentlichen oder halböffentlichen 
Raum platziert sind. Zudem besitzen sie eine Benutzerschnittstelle -- sehr oft in Form
eines Touchscreens. Sie bieten dabei in der Regel Zugang zu Informationen 
oder elektronischen Transaktionen \cite{retailing}.\\

Der Begriff \emph{Kiosk} hat seinen Ursprung im Persischen und steht dort für
ein zeltartiges Gartenhaus oder eine Art Erker an orientalischen Palästen \cite{meyers}.
Allgemeiner steht er in der islamischen Baukunst für einen pavillonähnlichen Bau \cite{taschenlexikon}.
In der heutigen, allgemeinsprachlichen Definition versteht man unter einem Kiosk eine Verkaufsstelle
für Zeitungen und Zeitschriften \cite{taschenlexikon, meyers}.\\
In der Bedeutung hat unsere heutige Definition dabei die äußere Form, das Pavillonähnliche, 
des ursprünglichen Kiosks übernommen. Oft sind Kioske alleinstehende Häuschen, die in ihrer Form
an einen Pavillon erinnern. Hinzugekommen zur Bedeutung ist das Öffentlichzugängliche\todo{Schreibweise prüfen}. Während der Kiosk 
in seiner ursprünglichen Wortbedeutung für etwas steht was meist an einen Palast angegliedert ist -- und 
somit vermutlich nur beschränkt zugänglich ist -- verstehen wir heute unter dem Begriff einen Ort, dessen Zugang
für jeden Menschen gedacht ist. Darüber hinaus ist er in der Regel an belebten und gut zugänglichen Orten 
wie Straßen, Plätze und Sehenswürdigkeiten platziert \cite{multimediale}.\\
Dieser Teil der Wortbedeutung hat sich auf das informationstechnische Kiosksystem übertragen. Mit dem 
pavillonähnlichen Bau hat es nichts mehr zu tun, dafür aber mit der Tatsache, dass es etwas Zugängliches 
im öffentlichen Raum darstellt \cite{multimediale}.\\
\citeext{multimediale} beschreibt weitere Parallelen: die Art der Kunden und 
die Verweildauer. Zum einen gibt es die Laufkundschaft, die durch optische oder akustische Reize
zum Herantreten animiert werden, sowie die Kunden die gezielt und mit einer bestimmten Absicht an den Kiosk
oder das Kiosksystem herantreten. Die Verweildauer des Kunden ist in beiden Szenarien kurz, 
vergleicht man den Besuch in einem Kiosk mit dem in einem Kaufhaus, oder das Benutzen eines Kiosksystems
mit dem Benutzen der eigenen elektronischen Geräte \cite{multimediale}.\\
Kiosksysteme sind bekannte Systeme. Fasst man den Begriff weit, so ist beispielsweise auch der Geldautomat ein Kiosksystem.
Aber auch Ticketautomaten oder der Self-Ordering Kiosk, wie ihn McDonalds 2011 in Europa 
eingeführt hat \cite{mcdonalds}, sind bekannte Systeme und zum Teil nicht mehr wegzudenken. Oft trifft man in 
Geschäften oder öffentlichen Gebäuden auf Kiosksysteme, die spezifische Informationen und Transaktionen 
bereitstellen. Denkbar wäre ein Ausleih-Kiosk in einer Bibliothek oder ein Kiosk in einem Kaufhaus, welcher einen
Lageplan und Informationen über die Geschäfte bereithält.\\
Dabei zeigt sich auch: Kiosksysteme können sehr unterschiedliche Zwecke haben. \citeext{across} klassifizieren daher
Kiosksysteme in vier Kategorien:

\begin{enumerate}
\item Informations-Kioske
\item Werbe-Kioske
\item Service-Kioske
\item Entertainment-Kioske
\end{enumerate}

Mischformen sind denkbar und üblich. Tatsächlich kommen Kiosksysteme die nur in eine der genannten Kategorien fallen
eher selten vor.\\
Der Informations-Kiosk hat die Aufgabe kontextbezogene Informationen bereitzustellen. Benutzer*innen sind motiviert und
gehen eher zielgerichtet vor. 
Der Werbe-Kiosk hat die Aufgabe eine Firma oder ein Produkt in der Öffentlichkeit zu
bewerben. Benutzer*innen müssen animiert werden das System zu nutzen und werden beispielsweise durch ein ansprechendes
Design motiviert.
Der Service-Kiosk ist ähnlich dem Informations-Kiosk. Zusätzlich können Benutzer*innen Transaktionen über eine 
Input-Schnittstelle tätigen. Beispielsweise ein Ticket kaufen.
Der Entertainment-Kiosk hat außer der Unterhaltung der Benutzer*innen keine weitere Aufgabe. Denkbar wäre ein solches
System zum Beispiel in einem Wartebereich. 