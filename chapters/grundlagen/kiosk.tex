\section{Kiosksysteme}
\label{section:kiosk}

Wenn man in der Informationstechnologie von Kiosksystemen spricht,
sind damit meist zugängliche Computersysteme gemeint die im öffentlichen oder halböffentlichen 
Raum platziert sind. Zudem besitzen sie eine Benutzerschnittstelle, sehr oft in Form
eines Touchscreens. Sie bieten dabei in der Regel Zugang zu Informationen 
oder elektronischen Transaktionen \cite{retailing}.\\

Der Begriff \emph{Kiosk} hat seinen Ursprung im Persischen und steht dort für
ein zeltartiges Gartenhaus oder eine Art Erker an orientalischen Palästen \cite{meyers}.
Allgemeiner steht er in der islamischen Baukunst für einen pavillonähnlichen Bau \cite{taschenlexikon}.
In der heutigen, allgemeinsprachlichen Definition versteht man unter einem Kiosk eine Verkaufsstelle
für Zeitungen und Zeitschriften \cite{taschenlexikon, meyers}.\\
In der Bedeutung hat unsere heutige Definition dabei die äußere Form, das Pavillonähnliche, 
des ursprünglichen Kiosks übernommen. Oft sind Kioske alleinstehende Häuschen, die in ihrer Form
an einen Pavillon erinnern. Hinzugekommen zur Bedeutung ist das öffentlich Zugängliche. Während der Kiosk 
in seiner ursprünglichen Wortbedeutung für etwas steht was meist an einen Palast angegliedert ist -- und 
somit vermutlich nur beschränkt zugänglich ist -- verstehen wir heute unter dem Begriff einen Ort, dessen Zugang
für jeden Menschen gedacht ist. Darüber hinaus ist er in der Regel an belebten und gut zugänglichen Orten 
in Städten, wie Straßen, Plätze und an Sehenswürdigkeiten, platziert.\\
Dieser Teil der Wortbedeutung hat sich auf das informationstechnische Kiosksystem übertragen \cite{multimediale}. Mit dem 
pavillonähnlichen Bau hat es nichts mehr zu tun, dafür aber mit der Tatsache das es etwas Zugängliches 
im öffentlichen Raum darstellt. \citeauthor{multimediale} beschreibt weitere Parallelen: die Art der Kunden und 
die Verweildauer. Zum einen gibt es die Laufkundschaft, die durch optische oder akustische Reize
zum Herantreten animiert werden, sowie die Kunden die gezielt und mit einer bestimmten Absicht an den Kiosk
oder das Kiosksystem herantreten. Und auch die Verweildauer der Kunden beschreibt \citeauthor{multimediale} als eine 
Parallele: In beidem Szenarien ist sie eher kurz im Vergleich zum Besuch in einem Kaufhaus oder der Nutzung der 
eigenen elektronischen Geräte \cite{multimediale}.\\


