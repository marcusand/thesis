\chapter{Fazit und Ausblick}
\label{chap:fazit}

Wie gezeigt werden konnte, eignen sich Webtechnologien sehr gut für
die Entwicklung von Kiosksoftware. Dies hat sich durch die Entwicklung
der \shst{} in der Praxis bestätigt. Webtechnologien bieten viele Vorteile, die andere
Technologien nicht leisten können. Da der Begriff der Webtechnologien jedoch
sehr weit fasst und ein riesiges Ökosystem an Technologien existiert, müssen bei
der Wahl dieser, die Anforderungen gründlich geprüft und Technologien gegeneinander 
abgewägt werden. Dafür wurde der Begriff der Browsertechnologien eingeführt und sich
weiter bei der Entwicklung der Clientanwendung auf diese beschränkt. 
Dabei haben sich einige Grenzen aufgetan, welche teilweise gelöst werden konnten.\\
Manche Anforderungen, wie die Mulitlingualität oder die Erreichbarkeit des CMS über das Internet,
sind mit Webtechnologien leicht umzusetzen, da sie auch Anforderungen typischer Webanwendungen sind.
Technologien und Lösungen sind dafür bereits vorhanden und bewährt.\\
Bei der Verwendung reiner Browsertechnologien für die Clientanwendung haben sich einige Vorteile 
gezeigt. Eine solche Applikation ist stark plattforumunabhängig, da Browser oder Browser-Umgebungen
für praktisch jede Plattform existieren und eine Anwendung so leicht portiert werden kann. Auch lassen
sich sehr einfache Deployment-Strategien entwickeln. So wurde sich bei der \shst{} entschieden, die
Applikationsdateien auf einem Webserver zu hosten -- genauso wie man es auch bei einer klassischen % build Prozess?
Webanwendung machen würde. Dies bietet darüber hinaus den Vorteil, dass die Applikation über das Internet
im Browser aufgerufen und als Vorschau genutzt werden kann. Für die Präsentation dieser Anwendung als
Kioskapplikation auf einer klassischen Workstation, muss auf die vollständige Geschlossenheit geachtet werden.
Hierfür bietet sich die kiosk Option von Google Chrome und das Definieren eines Webmanifests an.\\
Durch Ausliefern der Applikationsdateien über einen Webserver ist allerdings eine Offline-Verfügbarkeit
erstmal nicht gegeben. Diese kann durch Implementieren einer Single-Page Applikation und 
Service-Workern weitestgehend hergestellt werden.
Vollständige Offline-Verfügbarkeit ist dadurch aber nicht erreicht, da von Nutzenden initiierte
Transaktionen über das Internet an einen Server erfolgen müssen. Um solche Anfragen lokal zwischenzuspeichern,
ehe sie bei Netzwerkverfügbarkeit an einen Server erfolgen, sind weitere Technologien und 
Architektur-Pattern nötig, welche im Rahmen dieser Arbeit aber nicht weiter verfolgt wurden.\\
Die größte Herausforderung bei der Verwendung von Browsertechnologien stellte die Kommunikation 
der Anwendung mit Hardwaregeräten da. 
Die Möglichkeiten sind dabei auf die der Web APIs beschränkt. Diese lassen zwar die Kommunikation zu manchen
Geräten wie Aufnahme- und Ausgabequellen für Audio oder Webcams zu, für andere Geräte muss die Kommunikation
jedoch auf anderem Wege hergestellt werden. Im Rahmen dieser Arbeit wurde dafür die meso-connect Konvention 
entwickelt, welche über das MQTT/Websocket-Protokoll und eine standardisierte Kommunikation die Verbindung
zu unterschiedlichen Geräten möglich macht. Die Systemarchitektur für eine solche Kommunikation ist dabei zwar
aufwendig, die softwareseitige Anbindung aber ist, durch die strikte Konvention und die Implementierung der Konvention
in eine Softwarebibliothek, abstrahiert und daher eher simpel. Die entwickelte Konvention lässt sich darüber 
hinaus auch für andere Kiosksysteme und andere Projekte anwenden.\\
Es ist zu erwarten, dass die Web APIs kontinuierlich weiterentwickelt werden und in der Zukunft andere,
möglicherweise simplere, Lösungen zur Anbindung von Hardwaregeräten an Browseranwendungen denkbar sind. Eine Bemühung
stellt die WebUSB API dar \cite{web-usb}. Diese soll die direkte Kommunikation von Browseranwendungen mit angeschlossenen
USB Geräten ermöglichen. Diese Technologien steht allerdings noch am Anfang und ist nur in wenigen Browsern 
nutzbar. Zukünftig könnte sie allerdings eine Alternative für manche Anwendungsfälle darstellen.\\

Die so entstandenen Ergebnisse dieser Arbeit, sowie die Beschreibung der Architektur und der verwendeten
Softwaretechnologien der \shst{}, können ein Leitfaden für andere Softwareentwickler sein, die 
vor der Aufgabe stehen eine Kiosksoftware zu entwickeln. Dabei soll klar geworden sein inwiefern 
Webtechnologien sich für eine solche Aufgabe eignen und was dabei die Möglichkeiten und Grenzen sind.\\
Die Entwicklung der \shst{} ist für \meso{} die Grundlage für ein Produkt, 
welches in der Zukunft weiteren Kunden angeboten werden soll. Es sollen dabei in erster Linie Kunden 
aus dem Museums- und Ausstellungsbetrieb angesprochen werden.\\
Die meso-connect Konvention soll zukünftig nicht nur Anwendung bei weiteren Kiosksystemen finden, sondern 
möglichst bei allen Projekten der Firma \meso{}, bei denen eine MQTT Kommunikation vorgesehen ist, eingesetzt
werden.
