\section{Nicht-funktionale Anforderungen}
\label{sec:nicht-funktionale}

\begin{figure}
    \centering
    \includestandalone[width=1\textwidth]{figures/plant/out/schablone-umgebung}
    \caption{Umgebungsschablone für nicht-funktionale Anforderungen nach \citeext{requirements}}
    \label{fig:schablone-umgebung}
\end{figure}

Auch für die Formulierung der nicht-funktionalen Anforderungen kann wieder eine Schablone angelegt werden.
\citeext{requirements} definieren hierfür die Eigenschafts-, Umgebungs- und Prozessschablone. An dieser Stelle
können alle Anforderungen mit der Umgebungsschablone beschrieben werden, die anderen zwei werden deshalb nicht
weiter erläutert.

\begin{enumerate}[label=NFA\arabic*]
	\item\label{nfa1} Die Clientsoftware muss so gestaltet sein, dass sie mit Hardwarekomponenten kommunizieren kann.
	\item\label{nfa2} Die Clientsoftware sollte so gestaltet sein, dass sie offline zu bedienen ist. 
  \item\label{nfa3} Die Oberfläche der Clientsoftware sollte in sich geschlossen sein.\\
  Verlassen nicht möglich, Kein Laden zwischen den Seiten, Keine Internet-Links
  \item\label{nfa4} Die Oberfläche der Clientsoftware sollte so gestaltet sein, dass sie einfach 
  zu bedienen und Touch-friendly ist.
  \item\label{nfa5} Die Oberfläche der Clientsoftware muss multilingual sein.
  \item\label{nfa6} Die Clientsoftware sollte plattformunabhängig sein.
  \item\label{nfa7} Das Deployment der Clientsoftware sollte möglichst einfach und von außerhalb möglich sein.
  \item\label{nfa8} Die Software sollte modular und erweiterbar gehalten sein.
  \item\label{nfa9} Das CMS sollte über das Internet aufrufbar sein.\\
  Online-Vorschau
\end{enumerate}
