\section{Nicht-funktionale Anforderungen}
\label{sec:nicht-funktionale}

\begin{figure}
    \centering
    \includestandalone[width=1\textwidth]{figures/plant/out/schablone-umgebung}
    \includestandalone[width=1\textwidth]{figures/plant/out/schablone-eigenschaft}
    \caption{Umgebungs- und Eigenschaftsschablone für nicht-funktionale Anforderungen nach \citeext{requirements}}
    \label{fig:schablone-nicht-funktional}
\end{figure}

Die Formulierung nicht-funktionaler Anforderungen anhand einer Schablone fällt schwer. Nicht-funktionale
Anforderungen sind in der Literatur nicht einheitlich definiert und in ihrer Art unschärfer als funktionale
Anforderungen. Trotzdem schlagen \citeext{requirements} drei verschiedene Schablonen für die Formulierung
vor: Die Umgebungs-, Eigenschafts- und Prozessschablone. \autoref{fig:schablone-nicht-funktional} zeigt
die ersten beiden dieser Schablonen. Sie ähneln dabei der Schablone in \autoref{fig:schablone-funktion}.
Wieder spielt das Wichtigkeit-Schlüsselwort eine große Rolle. Anders als vorher ist das Subjekt nun 
nicht mehr \emph{<System>} sondern \emph{<Betrachtungsgegenstand>}, welcher im Falle der Umgebungsschablone
im Hauptsatz auf einen Teilbestandteil beschränkt werden kann.\\

Die Umgebungsschablone bildet Anforderungen ab, die von der Umgebung 
des Betrachtungsgegenstands abhängig sind. \ref{nfa1} und \ref{nfa2} sind nach dieser Schablone gebildet.
\ref{nfa2} beispielsweise fordert, dass die Software auch dann funktioniert wenn das Netz ausfällt in dem 
sich diese befindet. Das Netz ist dabei Teil der Umgebung und nicht Teil des Betrachtungsgegenstands.\\
Alle weiteren Anforderungen folgen grob der Struktur der Eigenschaftsschablone. Eigentlich fordert die 
Schablone immer einen definierten Wert für die beschriebene Eigenschaft, allerdings fordern nicht-funktionale
Anforderungen auch oft nur die reine Existenz einer Eigenschaft, wie z.B. in \ref{nfa3}.
In solchen Fällen schiebt sich die Eigenschaft hinter das 
Wichtigkeit-Schlüsselwort und der Vergleichsoperator und der Wert werden weggelassen.\\
Die Prozessschablone stellt Anforderungen an Akteure, z.B. an den Auftragnehmer, und nicht an das System.
Da in diesem Falle nur Anforderungen an die Softwaretechnologien betrachtet werden, findet diese Schablone 
keine Anwendung und ist deshalb auch nicht abgebildet.

\begin{enumerate}[label=\textbf{NFA\arabic*}]
	\item\label{nfa1} \textbf{Die Clientsoftware muss so gestaltet sein, dass sie zusammen mit Hardwarekomponenten 
  betrieben werden kann.}\\
  Für viele Kiosksysteme sind angeschlossene Hardwarekomponenten, als Nutzerschnittstelle oder Ausgabegeräte, essenziell.
  Gerade Servie-Kioske sind meist ohne ihre Hardwarekomponenten nicht zu benutzen. Das wohl eindrücklichste
  Beispiel hierfür ist der Ticketautomat: Ohne Kreditkartenterminal und Ticketdrucker wäre dieses Geräte nicht denkbar.\\
  Auch die \shst{} soll Hardwarekomponenten beinhalten. Für die Foto-Commitment-Funktion (\ref{fa4}) wird eine
  Kamera und für die Spenden-Funktion (\ref{fa3}) ein Kreditkartenterminal und ein Münzeinwurf benötigt. Alle 
  Geräte müssen mit der Clientsoftware kommunizieren können um eine Steuerung und ein Feedback über die Oberfläche
  zu gewährleisten. 
	\item\label{nfa2} \textbf{Die Clientsoftware sollte so gestaltet sein, dass sie auch offline 
  betrieben werden kann.}\\
  Die Bedienbarkeit der Clientsoftware sollte auch dann gewährleistet werden, wenn das Netz zwischenzeitlich
  ausfällt oder hoch belastet ist. Gerade Werbe- und Informations-Kioske sind sehr darauf bedacht ihre
  Nutzer möglichst lange am Gerät zu halten \cite{across}. Ein zu langer Ladevorgang durch welchen die 
  Oberfläche zwischenzeitlich nicht bedienbar ist, würde dazu führen, dass viele Nutzer sich 
  vom Gerät entfernen. Dies soll verhindert werden.\\
  Bei der \shst{} soll das Navigieren durch die Oberfläche und das Ausführen möglichst aller Funktionen 
  auch offline möglich sein. Lediglich Transaktionen bei denen der Nutzer Daten abschickt, wie Speichern 
  eines Fotos oder Eintragen für den Newsletter, sollen auch nur dann positives Feedback geben wenn 
  diese erfolgreich über das Netz erfolgt sind. Bei Nichterreichen eines Servers oder Services soll eine entsprechende
  Fehlermeldung angezeigt werden, welche den Nutzer darauf hinweist es später noch einmal zu versuchen.
  \item\label{nfa3} \textbf{Die Oberfläche der Clientsoftware muss multilingual sein.}\\
  Durch die meist öffentliche- oder halböffentliche Platzierung von Kiosksystemen, stehen diese
  immer einer großen Anzahl unterschiedlicher Nutzer*innen zur Verfügung. Um möglichst vielen
  ein optimales Erlebnis zu gewährleisten ist die Möglichkeit einer Spracheinstellung von großer Wichtigkeit.\\
  Auf Grund der typischen Besucher*innen und den finanziellen Möglichkeiten des Museums, werden die 
  Spracheinstellungen bei der \shst{} auf Deutsch und Englisch beschränkt.
  \item\label{nfa4} \textbf{Die Oberfläche der Clientsoftware sollte in sich geschlossen sein.}\\
  Verlassen nicht möglich, Kein Laden zwischen den Seiten, Keine Internet-Links
  \item\label{nfa5} \textbf{Die Oberfläche der Clientsoftware sollte einfach zu bedienen und Touch-friendly sein.}\\
  \item\label{nfa6} \textbf{Die Clientsoftware sollte plattformunabhängig sein.}\\
  Diese Anforderung ist nicht unbedingt typisch für Kiosksoftware im Allgemeinen, bei Kiosksoftware im Ausstellungs- und
  Messebetrieb wird sie jedoch häufig gestellt. Beispielsweise soll eine Software bei einer Messe auf einem
  großen Touchscreen und bei der nächsten auf mehreren iPads laufen. Oder eine Archivsoftware, welche als Kiosksystem
  in einer Ausstellung betrieben wird, soll nach der Ausstellung online zur Verfügung gestellt werden.\\ 
  Wie in \autoref{subsection:sharing-station-produkt} bereits erwähnt, soll die \shst{} als Plattform gedacht und
  unter den Gesichtspunkten der Adaptier- und Wiederverwendbarkeit entwickelt werden. Dazu zählt auch, dass die Software
  auf verschiedene Plattformen lauffähig ist, da zukünftige Kunden die Anforderung nach einer anderen Plattform
  haben könnten. 
  \item\label{nfa7} \textbf{Das Deployment der Clientsoftware sollte möglichst einfach und von außerhalb möglich sein.}\\
  \item\label{nfa8} \textbf{Die Software sollte modular und erweiterbar gehalten sein.}\\
  \item\label{nfa9} \textbf{Das CMS sollte über das Internet aufrufbar sein.}\\
  Online-Vorschau
\end{enumerate}
