\section{Nicht-funktionale Anforderungen}
\label{sec:nicht-funktionale}

\begin{figure}
    \centering
    \includestandalone[width=1\textwidth]{figures/plant/out/schablone-umgebung}
    \includestandalone[width=1\textwidth]{figures/plant/out/schablone-eigenschaft}
    \caption{Umgebungs- und Eigenschaftsschablone für nicht-funktionale Anforderungen nach \citeext{requirements}}
    \label{fig:schablone-nicht-funktional}
\end{figure}

Die Formulierung nicht-funktionaler Anforderungen anhand einer Schablone fällt schwer. Nicht-funktionale
Anforderungen sind in der Literatur nicht einheitlich definiert und in ihrer Art unschärfer als funktionale
Anforderungen. Trotzdem schlagen \citeext{requirements} drei verschiedene Schablonen für die Formulierung
vor: Die Umgebungs-, Eigenschafts- und Prozessschablone. \autoref{fig:schablone-nicht-funktional} zeigt
die ersten beiden dieser Schablonen. Sie ähneln dabei der Schablone in \autoref{fig:schablone-funktion}.
Wieder spielt das Wichtigkeit-Schlüsselwort eine große Rolle. Anders als vorher ist das Subjekt nun 
nicht mehr \emph{<System>} sondern \emph{<Betrachtungsgegenstand>}, welcher im Falle der Umgebungsschablone
im Hauptsatz auf einen Teilbestandteil beschränkt werden kann.\\

Die Umgebungsschablone bildet Anforderungen ab, die von der Umgebung 
des Betrachtungsgegenstands abhängig sind. \ref{nfa1} und \ref{nfa2} sind nach dieser Schablone gebildet.
\ref{nfa2} beispielsweise fordert, dass die Software auch dann funktioniert wenn das Netz ausfällt in dem 
sich diese befindet. Das Netz ist dabei Teil der Umgebung und nicht Teil des Betrachtungsgegenstands.\\
Alle weiteren Anforderungen folgen grob der Struktur der Eigenschaftsschablone. Eigentlich fordert die 
Schablone immer einen definierten Wert für die beschriebene Eigenschaft, allerdings fordern nicht-funktionale
Anforderungen auch oft nur die reine Existenz einer Eigenschaft, wie z.B. in \ref{nfa3}.
In solchen Fällen schiebt sich die Eigenschaft hinter das 
Wichtigkeit-Schlüsselwort und der Vergleichsoperator und der Wert werden weggelassen.\\
Die Prozessschablone stellt Anforderungen an Akteure, z.B. an den Auftragnehmer, und nicht an das System.
Da in diesem Falle nur Anforderungen an die Softwaretechnologien betrachtet werden, findet diese Schablone 
keine Anwendung und ist deshalb auch nicht abgebildet.

\begin{enumerate}[label=\textbf{NFA\arabic*}]
	\item\label{nfa1} \textbf{Die Clientsoftware muss so gestaltet sein, dass sie zusammen mit Hardwarekomponenten 
  betrieben werden kann.}
	\item\label{nfa2} \textbf{Die Clientsoftware sollte so gestaltet sein, dass sie auch offline 
  betrieben werden kann.} 
  \item\label{nfa3} \textbf{Die Oberfläche der Clientsoftware muss multilingual sein.}
  \item\label{nfa4} \textbf{Die Oberfläche der Clientsoftware sollte in sich geschlossen sein.}\\
  Verlassen nicht möglich, Kein Laden zwischen den Seiten, Keine Internet-Links
  \item\label{nfa5} \textbf{Die Oberfläche der Clientsoftware sollte einfach zu bedienen und Touch-friendly sein.}
  \item\label{nfa6} \textbf{Die Clientsoftware sollte plattformunabhängig sein.}
  \item\label{nfa7} \textbf{Das Deployment der Clientsoftware sollte möglichst einfach und von außerhalb möglich sein.}
  \item\label{nfa8} \textbf{Die Software sollte modular und erweiterbar gehalten sein.}
  \item\label{nfa9} \textbf{Das CMS sollte über das Internet aufrufbar sein.}\\
  Online-Vorschau
\end{enumerate}
