\section{Nicht-funktionale Anforderungen}
\label{sec:nicht-funktionale}

\begin{figure}
    \centering
    \includestandalone[width=1\textwidth]{figures/plant/out/schablone-umgebung}
    \includestandalone[width=1\textwidth]{figures/plant/out/schablone-eigenschaft}
    \caption{Umgebungs- und Eigenschaftsschablone für nicht-funktionale Anforderungen nach \citeext{requirements}}
    \label{fig:schablone-nicht-funktional}
\end{figure}

Die Formulierung nicht-funktionaler Anforderungen anhand einer Schablone fällt schwer. Nicht-funktionale
Anforderungen sind in der Literatur nicht einheitlich definiert und in ihrer Art unschärfer als funktionale
Anforderungen. Trotzdem schlagen \citeext{requirements} drei verschiedene Schablonen für die Formulierung
vor: Die Umgebungs-, Eigenschafts- und Prozessschablone. \autoref{fig:schablone-nicht-funktional} zeigt
die ersten beiden dieser Schablonen. Sie ähneln dabei der Schablone in \autoref{fig:schablone-funktion}.
Wieder spielt das Wichtigkeit-Schlüsselwort eine große Rolle. Anders als vorher ist das Subjekt nun 
nicht mehr \emph{<System>} sondern \emph{<Betrachtungsgegenstand>}, welcher im Falle der Umgebungsschablone
im Hauptsatz auf einen Teilbestandteil beschränkt werden kann (\emph{[Komponente des +]}).\\
Die Umgebungsschablone bildet Anforderungen ab, die von der Umgebung 
des Betrachtungsgegenstands abhängig sind. \ref{nfa1} und \ref{nfa2} sind nach dieser Schablone gebildet.
\ref{nfa2} beispielsweise fordert, dass die Software auch dann funktioniert wenn das Netz ausfällt in dem 
sich diese befindet. Das Netz ist dabei Teil der Umgebung und nicht Teil des Betrachtungsgegenstands.\\
Alle weiteren Anforderungen folgen grob der Struktur der Eigenschaftsschablone. Eigentlich fordert die 
Schablone immer einen definierten Wert für die beschriebene Eigenschaft, allerdings fordern nicht-funktionale
Anforderungen auch oft nur die reine Existenz einer Eigenschaft, wie z.B. in \ref{nfa3}.
In solchen Fällen schiebt sich die Eigenschaft hinter das 
Wichtigkeit-Schlüsselwort und der Vergleichsoperator und der Wert werden weggelassen.\\
Die dritte Schablone nach \citeext{requirements}, die Prozessschablone, ist in \autoref{fig:schablone-nicht-funktional}
nicht abgebildet. Sie stellt Anforderungen an Akteure, z.B. an den Auftragnehmer, und nicht an das System.
Da im Falle dieser Arbeit nur Anforderungen an die Softwaretechnologien betrachtet werden, findet diese Schablone 
keine Anwendung und wird nicht weiter erläutert.\\

Wie schon bei den funktionalen Anforderungen, folgt auch hier jeder Anforderung eine frei formulierte Erklärung. 
Diese bezieht sich immer zuerst auf Kiosksoftware und Kiosksysteme im Allgemeinen und erklärt dann die Bedeutung
der Anforderung für die \shst{}.

\begin{enumerate}[label=\textbf{NFA\arabic*}]
	\item\label{nfa1} \textbf{Die Clientsoftware muss so gestaltet sein, dass sie zusammen mit Hardwarekomponenten 
  betrieben werden kann.}\\
  Für viele Kiosksysteme sind angeschlossene Hardwarekomponenten, als Nutzerschnittstelle oder Ausgabegeräte, essenziell.
  Gerade Servie-Kioske sind meist ohne ihre Hardwarekomponenten nicht zu benutzen. Das wohl eindrücklichste
  Beispiel hierfür ist der Ticketautomat: Ohne Kreditkartenterminal und Ticketdrucker wäre dieses Geräte nicht denkbar.\\
  Auch die \shst{} soll Hardwarekomponenten beinhalten. Für die Foto-Commitment-Funktion (\ref{fa4}) wird eine
  Kamera und für die Spenden-Funktion (\ref{fa3}) ein Kreditkartenterminal und ein Münzeinwurf benötigt. Alle 
  Geräte müssen mit der Clientsoftware kommunizieren können um eine Steuerung und ein Feedback über die Oberfläche
  zu gewährleisten. 
	\item\label{nfa2} \textbf{Die Clientsoftware sollte so gestaltet sein, dass sie auch offline 
  betrieben werden kann.}\\
  Die Bedienbarkeit der Clientsoftware sollte auch dann gewährleistet werden, wenn das Netz zwischenzeitlich
  ausfällt oder hoch belastet ist. Gerade Werbe- und Informations-Kioske sind sehr darauf bedacht ihre
  Nutzer möglichst lange am Gerät zu halten \cite{across}. Ein zu langer Ladevorgang durch welchen die 
  Oberfläche zwischenzeitlich nicht bedienbar ist, würde dazu führen, dass viele Nutzer sich 
  vom Gerät entfernen. Dies soll verhindert werden.\\
  Bei der \shst{} soll das Navigieren durch die Oberfläche und das Ausführen möglichst aller Funktionen 
  auch offline möglich sein. Lediglich Transaktionen bei denen die Nutzer*innen Daten abschicken, wie Speichern 
  eines Fotos oder Eintragen für den Newsletter, sollen auch nur dann positives Feedback geben wenn 
  diese erfolgreich über das Netz erfolgt sind. Bei Nichterreichen eines Servers oder Services soll eine entsprechende
  Fehlermeldung angezeigt werden, welche den Nutzer darauf hinweist es später noch einmal zu versuchen.
  \item\label{nfa3} \textbf{Die Oberfläche der Clientsoftware muss multilingual sein.}\\
  Durch die meist öffentliche- oder halböffentliche Platzierung von Kiosksystemen, stehen diese
  immer einer großen Anzahl unterschiedlicher Nutzer*innen zur Verfügung. Um möglichst vielen
  ein optimales Erlebnis zu gewährleisten ist die Möglichkeit einer Spracheinstellung von großer Wichtigkeit.\\
  Auf Grund der typischen Besucher*innen und den finanziellen Möglichkeiten des Museums, werden die 
  Spracheinstellungen bei der \shst{} auf Deutsch und Englisch beschränkt.
  \item\label{nfa4} \textbf{Die Oberfläche der Clientsoftware muss in sich geschlossen sein.}\\
  Diese Anforderung ist typisch für Kiosksoftware. Sie bedeutet, dass die Oberfläche der Anwendung von Benutzer*innen 
  nicht verlassen werden kann. Sie soll den Eindruck eines geschlossenen Systems vermitteln. Dies ist zum eine
  aus der User-Experience Sicht wünschenswert aber auch aus einer sicherheitstechnischen Sicht relevant. Den 
  Benutzer*innen soll keine Möglichkeit geboten werden Schaden auf dem System anzurichten oder andere Webseiten
  im Netz zu erreichen.\\
  Gerade bei der Clientsoftware der \shst{}, welche selbst eine Webanwendung sein wird, muss hier besonderes darauf
  geachtet werden, dass keine anderen Webseiten erreicht werden können. Und ebenso das Verlassen dieser sollte
  nicht möglich sein. \todo{Andere Webseiten erreichen vielleicht raus}
  \item\label{nfa5} \textbf{Die Oberfläche der Clientsoftware sollte einfach zu bedienen und Touch-friendly sein.}\\
  \todo{weglassen?}
  \item\label{nfa6} \textbf{Die Clientsoftware sollte plattformunabhängig sein.}\\
  Diese Anforderung ist nicht unbedingt typisch für Kiosksoftware im Allgemeinen, bei Kiosksoftware im Ausstellungs- und
  Messebetrieb wird sie jedoch häufig gestellt. Beispielsweise soll eine Software bei einer Messe auf einem
  großen Touchscreen und bei der nächsten auf mehreren iPads laufen. Oder eine Archivsoftware, welche als Kiosksystem
  in einer Ausstellung betrieben wird, soll nach der Ausstellung online zur Verfügung gestellt werden. Die Software
  muss also so entwickelt werden, dass sie leicht auf ein anderes System übertragen werden kann. \\ 
  Wie in \autoref{subs:sharing-station-produkt} bereits erwähnt, soll die \shst{} als Produkt gedacht und
  unter den Gesichtspunkten der Adaptier- und Wiederverwendbarkeit entwickelt werden. Dazu zählt auch, dass die Software
  auf verschiedenen Plattformen lauffähig ist, da zukünftige Kunden die Anforderung nach einer anderen Plattform
  haben könnten. 
  \item\label{nfa7} \textbf{Das Deployment der Clientsoftware sollte möglichst einfach und von außerhalb möglich sein.}\\
  Bei Kiosksystemen ist es meist von Vorteil wenn Softwareupdates nicht vor Ort am Gerät eingespielt werden 
  müssen sondern von außerhalb erfolgen können.\\
  Bei der \shst{} ist dies keine vom Kunden geforderte Anforderung. Allerdings zeigt die Erfahrung, dass nachträgliche
  Anpassung und Fehlbehebung an der Software in den meisten Fällen nötig ist. Eine entsprechende Deployment-Strategie
  ist also von großem Vorteil.
  \item\label{nfa8} \textbf{Die Software sollte modular und erweiterbar gehalten sein.}\\
  Auch diese Anforderung ist nicht unbedingt typisch für Kiosksoftware aber dennoch ist es denkbar,
  dass sie häufig gestellt wird. 
  Eine modulare Software ist oft von Vorteil auch wenn sie meist längere Zeit in der Entwicklung
  benötigt. So können Teile der Software leichter gegen andere ersetzt oder neue zum System hinzugefügt 
  werden.\\
  Im Falle der \shst{} ist dies eine gewünschte Anforderung. Wie in \autoref{subs:sharing-station-produkt} beschrieben,
  soll die \shst{} zukünftig als Produkt angeboten werden und muss so in Teilen ersetzbar
  und leicht um weitere Funktionalitäten erweiterbar sein. 
  \item\label{nfa9} \textbf{Das CMS muss über das Internet aufrufbar sein.}\\
  Ebenso wie bei dem Deployment ist es auch bei der Inhaltserstellung und -pflege von Vorteil wenn diese bei 
  Kiosksystemen nicht vor Ort am Gerät oder im lokalen Netz erfolgen muss, sondern von außerhalb gemacht 
  werden kann. Das CMS ist dafür am besten ein über das Internet erreichbarer Dienst.\\
  Diese Anforderung gilt genauso für die \shst{}. Alleine schon da die Büroräume des Kunden sich an einem 
  anderen Ort wie das Museum befinden, ist dies eine verpflichtende Anforderung.
\end{enumerate}
