\section{Funktionale Anforderungen}
\label{sec:funktionale}

\begin{figure}
    \centering
    \includestandalone[width=1\textwidth]{figures/plant/out/schablone-funktion}
    \caption{Schablone für funktionale Anforderungen nach \citeext{requirements}}
    \label{fig:schablone-funktion}
\end{figure}

Alle Anforderungen sind in natürlicher Sprache verfasst und folgen dabei einer bestimmten Syntax. Diese
Syntax folgt den Schablonenregeln von \citeext{requirements}. \autoref{fig:schablone-funktion} zeigt die
Schablone für die Formulierung einer funktionalen Anforderung. \emph{System} entspricht dem 
Subjekt, z.B. die \shst{} oder die Clientanwendung. Anschließend wird die \emph{Wichtigkeit} festgelegt. 
Das Schlüsselwort \emph{muss} bedeutet, eine Anforderung ist im rechtlichen Sinne verpflichtend. \emph{Sollte} 
beschreibt eine Anforderung, welche nicht verpflichtend ist aber die Zufriedenheit erhöht. Und \emph{wird}
wird verwendet, um Anforderungen zu definieren, welche in der Zukunft erst integriert werden.\\
Die \emph{Art der Funktionalität} beschreibt die unterschiedlichen Systemaktivitäten. Wird hier kein Schlüsselwort
eingesetzt, so beschreibt die Anforderung eine selbständige Aktivität. Der Satzteil \emph{die Möglichkeit bieten}
setzt immer eine Benutzerinteraktion voraus. \emph{Fähig sein} definiert eine Schnittstellenanforderung.\\
Schließlich folgt das \emph{Objekt}, für welches die Funktionalität gefordert wird und das so genannte \emph{Prozesswort}. Das
Prozesswort ist das Verb des Satzes, welches die Funktionalität identifiziert.\\
Gilt eine Anforderung nur unter einer bestimmten Bedingung, wird diese dem Wichtigkeit-Schlüsselwort vorausgestellt und
das System-Schlüsselwort wird vor der Art der Funktionalität platziert. Ein Beispiel hierfür ist \ref{fa8}.\\

Den folgenden Anforderungen folgt jeweils eine kurze Erklärung, welche nicht der eben beschriebenen Syntax folgt,
sondern lediglich die Anforderung weiter ausführen soll. 

\begin{enumerate}[label=\textbf{FA\arabic*}]
	\item\label{fa1} \textbf{Die \shst{} muss den Besuchenden die Möglichkeit bieten, Informationen über das Museum zu erhalten.}\\
  Diese Informationen sollen dabei in Artikelform und zu unterschiedlichen Themen abrufbar sein. Ein Menü soll Überblick über 
  die verschiedenen Artikel geben. 
	\item\label{fa2} \textbf{Die \shst{} muss den Besuchenden die Möglichkeit bieten, Informationen über andere Projekte zu erhalten.}\\
  Analog zu den Informationen über das Museum, sollen Informationen über andere, ähnliche Projekte abrufbar sein. Diese sollen dabei
  in einem eigenen Menü organisiert sein.
  \item\label{fa3} \textbf{Die \shst{} muss den Besuchenden die Möglichkeit bieten, eine Spende zu hinterlassen.}\\
  Das Spenden soll dabei kontaktlos per Kreditkarte, EC-Karte und Mobile Payment oder mit Bargeld, in Form eines Münzeinwurfs, möglich sein.
  \item\label{fa4} \textbf{Die \shst{} muss den Besuchenden die Möglichkeit bieten, ein Foto-Commitment zu hinterlassen.}\\
  Diese Funktion soll ähnlich wie eine Fotobox funktionieren. Eine an das System angeschlossene Webcam soll die Möglichkeit bieten ein Foto 
  zu machen und dieses in einer Galerie zu speichern oder sich selbst per Mail zu schicken. Vor Ort sollen außerdem kleine Tafeln ausgelegt werden
  auf denen Besuchende ein soziales Commitment schreiben und dann auf dem Foto präsentieren können. 
  \item\label{fa5} \textbf{Die \shst{} sollte den Besuchenden die Möglichkeit bieten, sich für einen Newsletter einzutragen.}\\ % Schnittstellenanforderung?
  Das Eintragen soll über ein Formular und eine Bildschirmtastatur möglich sein. Der Request soll dabei an das Wordpress-basierte 
  Newslettersystem des Museums erfolgen. 
  \item\label{fa6} \textbf{Die \shst{} sollte den Besuchenden die Möglichkeit bieten, Informationen mitzunehmen.}\\
  Am Ende von Artikeln sollen weiterführende Informationen und Verweise auf Webseiten per QR-Code verlinkt sein. Besuchende können diese 
  Informationen so auf dem eigenen Geräten lesen und speichern. 
  \item\label{fa7} \textbf{Die \shst{} sollte den Mitarbeitenden des Museums die Möglichkeit bieten, die Inhalte über ein Content
  Management System (CMS) zu verwalten.}\\
  Die Inhalte, gerade die Artikel über das Museum und andere Projekte, sollen redaktionell verwaltbar sein. Dies soll über ein 
  intuitiv zu bedienendes CMS möglich sein. 
  \item\label{fa8} \textbf{Nach einer gewissen Zeit ohne Benutzerinteraktion, sollte die Oberfläche der Clientanwendung in einen
  Idle-Modus wechseln.}\\
  Dieser Modus soll wie ein Bildschirmschoner wirken. Eine Animation soll dabei die Oberfläche überdecken und mit Schlagwörtern 
  und bewegten Formen die Besuchenden zur Interaktion animieren. 
\end{enumerate}
