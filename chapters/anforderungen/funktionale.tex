\section{Funktionale Anforderungen}
\label{sec:funktionale}

\begin{figure}
    \centering
    \includestandalone[width=1\textwidth]{figures/plant/out/schablone-funktion}
    \caption{Schablone für funktionale Anforderungen nach \citeext{requirements}}
    \label{fig:schablone-funktion}
\end{figure}

Alle Anforderungen sind in natürlicher Sprache verfasst und folgen dabei einer bestimmten Syntax. Diese
Syntax folgt den Schablonenregeln von \citeext{requirements}. \autoref{fig:schablone-funktion} zeigt die
Schablone für die Formulierung einer funktionalen Anforderung. \emph{<System>} entspricht dem 
Subjekt\todo{Wirklich subjekt?} um welches
es geht, z.B. die \shst{} oder die Clientsoftware. Anschließend wird die Wichtigkeit festgelegt. Das 
Schlüsselwort \emph{muss} bedeutet, eine Anforderung ist im rechtlichen Sinne verpflichtend. \emph{Sollte} 
beschreibt eine Anforderung, welche nicht verpflichtend ist aber die Zufriedenheit erhöht. Und \emph{wird}
wird verwendet, um Anforderungen zu definieren, welche in der Zukunft erst integriert werden.\\
Die Art der Funktionalität beschreibt die unterschiedlichen Systemaktivitäten. Wird hier kein Schlüsselwort
eingesetzt, so beschreibt die Anforderung eine selbständige Aktivität. Der Satzteil \emph{die Möglichkeit bieten}
setzt immer eine Benutzerinteraktion voraus. \emph{Fähig sein} definiert eine Schnittstellenanforderung.\\
Schließlich folgt das Objekt, für welches die Funktionalität gefordert wird und das so genannte Prozesswort -- das
Verb des Satzes, welches die Funktionalität identifiziert.\\
Gilt eine Anforderung nur unter einer bestimmten Bedingung, wird diese dem Wichtigkeit-Schlüsselwort vorausgestellt und
das System-Schlüsselwort wird vor der Art der Funktionalität platziert. Ein Beispiel hierfür ist \ref{fa8}.

\begin{enumerate}[label=\textbf{FA\arabic*}]
	\item\label{fa1} \textbf{Das System muss den Besucher*innen die Möglichkeit bieten, Informationen über das Museum zu erhalten.}\\
  Die \shst{} soll über das Museum informieren. Die Besucher*innen sollen dabei verschiedene Artikel zu unterschiedlichen Themen lesen können. 
	\item\label{fa2} \textbf{Das System muss den Besucher*innen die Möglichkeit bieten, Informationen über andere Projekte zu erhalten.}\\
  Die \shst{} soll über andere ähnliche Projekte informieren. Andere Projekte sollen dabei auf eigenen Projektseiten vorgestellt werden.
  \item\label{fa3} \textbf{Das System muss den Besucher*innen die Möglichkeit bieten, eine Spende zu hinterlassen.}\\
  Die \shst{} soll Spenden entweder kontaktlos per Kreditkarte, EC-Karte, und Mobile Payment oder Bargeld, in Form eines Münzeinwurfs, akzeptieren.
  \item\label{fa4} \textbf{Das System muss den Besucher*innen die Möglichkeit bieten, ein Foto-Commitment zu hinterlassen.}\\
  \item\label{fa5} \textbf{Das System sollte den Besucher*innen die Möglichkeit bieten, sich für einen Newsletter einzutragen.}\\ % Schnittstellenanforderung?
  \item\label{fa6} \textbf{Das System sollte den Besucher*innen die Möglichkeit bieten, Informationen mitzunehmen.}\\
  \item\label{fa7} \textbf{Das System sollte den Mitarbeiter*innen des Museums die Möglichkeit bieten, die Inhalte über ein CMS zu verwalten.}\\
  \item\label{fa8} \textbf{Nach einer gewissen Zeit ohne Benutzerinteraktion, sollte die Oberfläche der Clientsoftware in einen
  Idle-Modus wechseln.}\\
\end{enumerate}


\ref{fa1} zeigt, dass ...
