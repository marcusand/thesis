\chapter{Anforderungsanalyse}
\label{chap:anforderungen}

Nun sind wir am Ausgangspunkt der Arbeit angelangt: Es soll die Software für ein Kiosksystem
entwickelt werden. Im Rahmen dieser Arbeit und am Beispiel der \shst{}
geschieht dies alleinig mit Webtechnologien. Um die Möglichkeiten und Grenzen dieser Technologien
strukturiert und genau prüfen zu können, werden in diesem Kapitel die Anforderungen an das System
und an die Software des Systems analysiert.
Dies ist auch nötig, da die Webtechnologien nun in einem neuen Kontext stehen. Anforderungen an diese
Technologien in einem klassischen Webentwicklungskontext sind klarer und bekannter. Im Kontext
des Kiosksystems bedarf es einer neuen und veränderten Betrachtung. 
Diese neuen und veränderten Anforderungen werden im folgenden Kapitel untersucht und strukturiert.\\

Nach einer üblichen Methode der Systemanalyse werden die Anforderungen in funktionale und 
nicht-funktionale Anforderungen unterteilt. Dabei definieren funktionale Anforderungen die 
reine Funktionalität und geben Antworten 
auf die Frage "Was soll das System machen?". Nicht-funktionale Anforderungen bilden die
restlichen Anforderungen ab und definieren dabei Anforderungen an die Eigenschaften und Qualitäten der
Funktionen \cite{systemanalyse}. Sie ergänzen dabei die funktionalen 
Anforderungen \cite{systematisches}. Nicht-funktionale Anforderungen werden oft noch weiter unterschieden,
beispielsweise in Qualitätsanforderungen und Randbedingungen \cite{systemanalyse,systematisches}.
Da sich der Anforderungsumfang in dieser Arbeit in Maßen hält, wird diese Unterscheidung im Folgenden
nicht vorgenommen.\\
Funktionale Anforderungen sind immer sehr spezifisch, da sie die konkreten Features
und Funktionalitäten abbilden. So beschreiben auch hier die funktionalen Anforderungen die konkreten
Eigenschaften der \shst{} und geben dabei einen guten Überblick über den geplanten Funktionsumfang.
Nicht-funktionale Anforderungen hingegen sind generischer und lassen sich leichter auf andere Projekte
übertragen. So sind die im Folgenden beschriebenen nicht-funktionalen Anforderungen, Anforderungen, die
auch typisch für Kiosksoftware im Allgemeinen sind. Sie sind für die Zielsetzung dieser Arbeit also von 
größerer Bedeutung.\\
Beim Definieren der Anforderungen beschränkt sich diese Arbeit auf die Anforderungen an die Software.
Dabei stellen manche Anforderungen auch Ansprüche an andere Teile des Systems. Diese Überschneidung
lässt sich nicht vermeiden, denn gerade die funktionalen Anforderungen stellen meist auch Anforderungen an das 
Gesamtsystem.\\ 

\section{Funktionale Anforderungen}
\label{sec:funktionale}

\begin{figure}
    \centering
    \includestandalone[width=1\textwidth]{figures/plant/out/schablone-funktion}
    \caption{Schablone für funktionale Anforderungen nach \citeext{requirements}}
    \label{fig:schablone-funktion}
\end{figure}

Alle Anforderungen sind in natürlicher Sprache verfasst und folgen dabei einer bestimmten Syntax. Diese
Syntax folgt den Schablonenregeln von \citeext{requirements}. \autoref{fig:schablone-funktion} zeigt die
Schablone für die Formulierung einer funktionalen Anforderung. \emph{System} entspricht dem 
Subjekt, z.B. die \shst{} oder die Clientsoftware. Anschließend wird die \emph{Wichtigkeit} festgelegt. 
Das Schlüsselwort \emph{muss} bedeutet, eine Anforderung ist im rechtlichen Sinne verpflichtend. \emph{Sollte} 
beschreibt eine Anforderung, welche nicht verpflichtend ist aber die Zufriedenheit erhöht. Und \emph{wird}
wird verwendet, um Anforderungen zu definieren, welche in der Zukunft erst integriert werden.\\
Die \emph{Art der Funktionalität} beschreibt die unterschiedlichen Systemaktivitäten. Wird hier kein Schlüsselwort
eingesetzt, so beschreibt die Anforderung eine selbständige Aktivität. Der Satzteil \emph{die Möglichkeit bieten}
setzt immer eine Benutzerinteraktion voraus. \emph{Fähig sein} definiert eine Schnittstellenanforderung.\\
Schließlich folgt das \emph{Objekt}, für welches die Funktionalität gefordert wird und das so genannte \emph{Prozesswort}. Das
Prozesswort ist das Verb des Satzes, welches die Funktionalität identifiziert.\\
Gilt eine Anforderung nur unter einer bestimmten Bedingung, wird diese dem Wichtigkeit-Schlüsselwort vorausgestellt und
das System-Schlüsselwort wird vor der Art der Funktionalität platziert. Ein Beispiel hierfür ist \ref{fa8}.\\

Den folgenden Anforderungen folgt jeweils eine kurze Erklärung, welche nicht der eben beschriebenen Syntax folgt,
sondern lediglich die Anforderung weiter ausführen soll. 

\begin{enumerate}[label=\textbf{FA\arabic*}]
	\item\label{fa1} \textbf{Die \shst{} muss den Besuchenden die Möglichkeit bieten, Informationen über das Museum zu erhalten.}\\
  Diese Informationen sollen dabei in Artikelform und zu unterschiedlichen Themen abrufbar sein. Ein Menü soll Überblick über 
  die verschiedenen Artikel geben. 
	\item\label{fa2} \textbf{Die \shst{} muss den Besuchenden die Möglichkeit bieten, Informationen über andere Projekte zu erhalten.}\\
  Analog zu den Informationen über das Museum, sollen Informationen über andere, ähnliche Projekte abrufbar sein. Diese sollen dabei
  in einem eigenen Menü organisiert sein.
  \item\label{fa3} \textbf{Die \shst{} muss den Besuchenden die Möglichkeit bieten, eine Spende zu hinterlassen.}\\
  Das Spenden soll dabei kontaktlos per Kreditkarte, EC-Karte und Mobile Payment oder mit Bargeld, in Form eines Münzeinwurfs, möglich sein.
  \item\label{fa4} \textbf{Die \shst{} muss den Besuchenden die Möglichkeit bieten, ein Foto-Commitment zu hinterlassen.}\\
  Diese Funktion soll ähnlich wie eine Fotobox funktionieren. Eine an das System angeschlossene Webcam, soll die Möglichkeit bieten ein Foto 
  zu machen und dieses in einer Galerie zu speichern oder sich selbst per Mail zu schicken. Vor Ort sollen außerdem kleine Tafeln ausgelegt werden
  auf denen Besuchende ein soziales Commitment schreiben und dann auf dem Foto präsentieren können. 
  \item\label{fa5} \textbf{Die \shst{} sollte den Besuchenden die Möglichkeit bieten, sich für einen Newsletter einzutragen.}\\ % Schnittstellenanforderung?
  Das Eintragen soll über ein Formular und eine Bildschirmtastatur möglich sein. Der Request soll dabei an das Wordpress-basierte 
  Newslettersystem des Museums erfolgen. 
  \item\label{fa6} \textbf{Die \shst{} sollte den Besuchenden die Möglichkeit bieten, Informationen mitzunehmen.}\\
  Am Ende von Artikeln sollen weiterführende Informationen und Verweise auf Webseiten per QR-Code verlinkt sein. Besuchende können diese 
  Informationen so auf dem eigenen Geräten lesen und speichern. 
  \item\label{fa7} \textbf{Die \shst{} sollte den Mitarbeitenden des Museums die Möglichkeit bieten, die Inhalte über ein Content
  Management System (CMS) zu verwalten.}\\
  Die Inhalte, gerade die Artikel über das Museum und andere Projekte, sollen redaktionell verwaltbar sein. Dies soll über ein 
  intuitiv zu bedienendes CMS möglich sein. 
  \item\label{fa8} \textbf{Nach einer gewissen Zeit ohne Benutzerinteraktion, sollte die Oberfläche der Clientsoftware in einen
  Idle-Modus wechseln.}\\
  Dieser Modus soll wie ein Bildschirmschoner wirken. Eine Animation soll dabei die Oberfläche überdecken und mit Schlagwörtern 
  und bewegten Formen die Besuchenden zur Interaktion animieren. 
\end{enumerate}

\section{Nicht-funktionale Anforderungen}
\label{sec:nicht-funktionale}

\begin{figure}
    \centering
    \includestandalone[width=1\textwidth]{figures/plant/out/schablone-umgebung}
    \includestandalone[width=1\textwidth]{figures/plant/out/schablone-eigenschaft}
    \caption{Umgebungs- und Eigenschaftsschablone für nicht-funktionale Anforderungen nach \citeext{requirements}}
    \label{fig:schablone-nicht-funktional}
\end{figure}

Die Formulierung nicht-funktionaler Anforderungen anhand einer Schablone fällt schwer. Nicht-funktionale
Anforderungen sind in der Literatur nicht einheitlich definiert und in ihrer Art unschärfer als funktionale
Anforderungen. Trotzdem schlagen \citeext{requirements} drei verschiedene Schablonen für die Formulierung
vor: Die Umgebungs-, Eigenschafts- und Prozessschablone. \autoref{fig:schablone-nicht-funktional} zeigt
die ersten beiden dieser Schablonen. Sie ähneln dabei der Schablone in \autoref{fig:schablone-funktion}.
Wieder spielt das Wichtigkeit-Schlüsselwort eine große Rolle. Anders als vorher ist das Subjekt nun 
nicht mehr \emph{<System>} sondern \emph{<Betrachtungsgegenstand>}, welcher im Falle der Umgebungsschablone
im Hauptsatz auf einen Teilbestandteil beschränkt werden kann (\emph{[Komponente des +]}).\\

Die Umgebungsschablone bildet Anforderungen ab, die von der Umgebung 
des Betrachtungsgegenstands abhängig sind. \ref{nfa1} und \ref{nfa2} sind nach dieser Schablone gebildet.
\ref{nfa2} beispielsweise fordert, dass die Software auch dann funktioniert wenn das Netz ausfällt in dem 
sich diese befindet. Das Netz ist dabei Teil der Umgebung und nicht Teil des Betrachtungsgegenstands.\\
Alle weiteren Anforderungen folgen grob der Struktur der Eigenschaftsschablone. Eigentlich fordert die 
Schablone immer einen definierten Wert für die beschriebene Eigenschaft, allerdings fordern nicht-funktionale
Anforderungen auch oft nur die reine Existenz einer Eigenschaft, wie z.B. in \ref{nfa3}.
In solchen Fällen schiebt sich die Eigenschaft hinter das 
Wichtigkeit-Schlüsselwort und der Vergleichsoperator und der Wert werden weggelassen.\\
Die dritte Schablone nach \citeext{requirements}, die Prozessschablone, ist in \autoref{fig:schablone-nicht-funktional}
nicht abgebildet. Sie stellt Anforderungen an Akteure, z.B. an den Auftragnehmer, und nicht an das System.
Da im Falle dieser Arbeit nur Anforderungen an die Softwaretechnologien betrachtet werden, findet diese Schablone 
keine Anwendung und wird nicht weiter erläutert. 

\begin{enumerate}[label=\textbf{NFA\arabic*}]
	\item\label{nfa1} \textbf{Die Clientsoftware muss so gestaltet sein, dass sie zusammen mit Hardwarekomponenten 
  betrieben werden kann.}\\
  Für viele Kiosksysteme sind angeschlossene Hardwarekomponenten, als Nutzerschnittstelle oder Ausgabegeräte, essenziell.
  Gerade Servie-Kioske sind meist ohne ihre Hardwarekomponenten nicht zu benutzen. Das wohl eindrücklichste
  Beispiel hierfür ist der Ticketautomat: Ohne Kreditkartenterminal und Ticketdrucker wäre dieses Geräte nicht denkbar.\\
  Auch die \shst{} soll Hardwarekomponenten beinhalten. Für die Foto-Commitment-Funktion (\ref{fa4}) wird eine
  Kamera und für die Spenden-Funktion (\ref{fa3}) ein Kreditkartenterminal und ein Münzeinwurf benötigt. Alle 
  Geräte müssen mit der Clientsoftware kommunizieren können um eine Steuerung und ein Feedback über die Oberfläche
  zu gewährleisten. 
	\item\label{nfa2} \textbf{Die Clientsoftware sollte so gestaltet sein, dass sie auch offline 
  betrieben werden kann.}\\
  Die Bedienbarkeit der Clientsoftware sollte auch dann gewährleistet werden, wenn das Netz zwischenzeitlich
  ausfällt oder hoch belastet ist. Gerade Werbe- und Informations-Kioske sind sehr darauf bedacht ihre
  Nutzer möglichst lange am Gerät zu halten \cite{across}. Ein zu langer Ladevorgang durch welchen die 
  Oberfläche zwischenzeitlich nicht bedienbar ist, würde dazu führen, dass viele Nutzer sich 
  vom Gerät entfernen. Dies soll verhindert werden.\\
  Bei der \shst{} soll das Navigieren durch die Oberfläche und das Ausführen möglichst aller Funktionen 
  auch offline möglich sein. Lediglich Transaktionen bei denen der Nutzer Daten abschickt, wie Speichern 
  eines Fotos oder Eintragen für den Newsletter, sollen auch nur dann positives Feedback geben wenn 
  diese erfolgreich über das Netz erfolgt sind. Bei Nichterreichen eines Servers oder Services soll eine entsprechende
  Fehlermeldung angezeigt werden, welche den Nutzer darauf hinweist es später noch einmal zu versuchen.
  \item\label{nfa3} \textbf{Die Oberfläche der Clientsoftware muss multilingual sein.}\\
  Durch die meist öffentliche- oder halböffentliche Platzierung von Kiosksystemen, stehen diese
  immer einer großen Anzahl unterschiedlicher Nutzer*innen zur Verfügung. Um möglichst vielen
  ein optimales Erlebnis zu gewährleisten ist die Möglichkeit einer Spracheinstellung von großer Wichtigkeit.\\
  Auf Grund der typischen Besucher*innen und den finanziellen Möglichkeiten des Museums, werden die 
  Spracheinstellungen bei der \shst{} auf Deutsch und Englisch beschränkt.
  \item\label{nfa4} \textbf{Die Oberfläche der Clientsoftware muss in sich geschlossen sein.}\\
  Diese Anforderung ist typisch für Kiosksoftware. Sie bedeutet, dass die Oberfläche der Anwendung von Benutzer*innen 
  nicht verlassen werden kann. Sie soll den Eindruck eines geschlossenen Systems vermitteln. Dies ist zum eine
  aus der User-Experience Sicht wünschenswert aber auch aus einer sicherheitstechnischen Sicht relevant. Den 
  Benutzer*innen soll keine Möglichkeit geboten werden Schaden auf dem System anzurichten oder andere Webseiten
  im Netz zu erreichen.\\
  Gerade bei der Clientsoftware der \shst{}, welche selbst eine Webanwendung sein wird, muss hier besonderes darauf
  geachtet werden, dass keine anderen Webseiten erreicht werden können. Aber ebenso das Verlassen dieser sollte
  nicht möglich sein.
  \item\label{nfa5} \textbf{Die Oberfläche der Clientsoftware sollte einfach zu bedienen und Touch-friendly sein.}\\

  \item\label{nfa6} \textbf{Die Clientsoftware sollte plattformunabhängig sein.}\\
  Diese Anforderung ist nicht unbedingt typisch für Kiosksoftware im Allgemeinen, bei Kiosksoftware im Ausstellungs- und
  Messebetrieb wird sie jedoch häufig gestellt. Beispielsweise soll eine Software bei einer Messe auf einem
  großen Touchscreen und bei der nächsten auf mehreren iPads laufen. Oder eine Archivsoftware, welche als Kiosksystem
  in einer Ausstellung betrieben wird, soll nach der Ausstellung online zur Verfügung gestellt werden. Die Software
  muss also so entwickelt werden, dass sie leicht auf ein anderes System übertragen werden kann. \\ 
  Wie in \autoref{ssub:sharing-station-produkt} bereits erwähnt, soll die \shst{} als Produkt gedacht und
  unter den Gesichtspunkten der Adaptier- und Wiederverwendbarkeit entwickelt werden. Dazu zählt auch, dass die Software
  auf verschiedene Plattformen lauffähig ist, da zukünftige Kunden die Anforderung nach einer anderen Plattform
  haben könnten. 
  \item\label{nfa7} \textbf{Das Deployment der Clientsoftware sollte möglichst einfach und von außerhalb möglich sein.}\\
  Bei Kiosksystemen ist es meist immer von Vorteil wenn Softwareupdates nicht vor Ort am Gerät eingespielt werden 
  müssen sondern von außerhalb erfolgen können.\\
  Bei der \shst{} ist dies keine vom Kunden geforderte Anforderung. Allerdings zeigt die Erfahrung, dass nachträgliche
  Anpassung und Fehlbehebung an der Software in den meisten Fällen nötig ist. Eine entsprechende Deployment Strategie
  ist also von großem Vorteil.
  \item\label{nfa8} \textbf{Die Software sollte modular und erweiterbar gehalten sein.}\\
  Auch diese Anforderung ist nicht unbedingt typisch für Kiosksoftware, aber doch denkbar, dass sie häufig gestellt wird. 
  Eine modulare und erweiterbare Software ist oft von Vorteil auch wenn sie meist längere Zeit in der Entwicklung
  benötigt.\\
  Im Falle der \shst{} ist dies eine gewünschte Anforderung. Wie in \autoref{ssub:sharing-station-produkt} beschrieben
  soll die \shst{} zukünftig als Produkt angeboten werden und muss so auch um weitere Funktionalitäten 
  erweitert werden können. 
  \item\label{nfa9} \textbf{Das CMS muss über das Internet aufrufbar sein.}\\
  Ebenso wie bei dem Deployment ist es auch bei der Inhaltserstellung und -pflege von Vorteil wenn diese bei 
  Kiosksystemen nicht vor Ort am Gerät oder im lokalen Netz erfolgen muss, sondern von außerhalb gemacht 
  werden kann. Das CMS ist dafür am besten ein über das Internet erreichbarer Dienst.\\
  Diese Anforderung gilt genauso für die \shst{}. Alleine schon da die Büroräume des Kunden sich an einem 
  anderen Ort wie das Museum befinden ist dies eine verpflichtende Anforderung.
\end{enumerate}
