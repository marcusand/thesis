\chapter{Anforderungsanalyse}
\label{chap:anforderungen}

Nun sind wir am Ausgangspunkt der Arbeit angelangt: Es soll die Software für ein Kiosksystem
entwickelt werden. Im Rahmen dieser Arbeit und am Beispiel der \shst{}
geschieht dies alleinig mit Webtechnologien. Um die Möglichkeiten und Grenzen dieser Technologien
strukturiert und genau prüfen zu können, werden in diesem Kapitel die Anforderungen an das System
und an die Software des Systems analysiert.
Dies ist auch nötig, da die Webtechnologien nun in einem neuen Kontext stehen. Anforderungen an diese
Technologien in einem klassischen Webentwicklungskontext sind klarer und bekannter. Im Kontext
des Kiosksystems bedarf es einer neuen und veränderten Betrachtung.\\

Nach einer üblichen Methode der Systemanalyse werden die Anforderungen in funktionale und 
nicht-funktionale Anforderungen unterteilt. Dabei definieren funktionale Anforderungen die 
reine Funktionalität und geben Antworten 
auf die Frage "Was soll das System machen?". Nicht-funktionale Anforderungen bilden die
restlichen Anforderungen ab und definieren dabei Anforderungen an die Eigenschaften und Qualitäten der
Funktionen \cite{systemanalyse}. Sie ergänzen dabei die funktionalen 
Anforderungen \cite{systematisches}. Nicht-funktionale Anforderungen werden oft noch weiter unterschieden,
beispielsweise in Qualitätsanforderungen und Randbedingungen \cite{systemanalyse,systematisches}.
Da sich der Anforderungsumfang in dieser Arbeit in Maßen hält, wird diese Unterscheidung im Folgenden
nicht vorgenommen.\\
Funktionale Anforderungen sind immer sehr spezifisch, da sie die konkreten Features
und Funktionalitäten abbilden. So beschreiben auch hier die funktionalen Anforderungen die konkreten
Eigenschaften der \shst{} und geben dabei einen guten Überblick über den geplanten Funktionsumfang.
Nicht-funktionale Anforderungen hingegen sind generischer und lassen sich leichter auf andere Projekte
übertragen. So sind die im Folgenden beschriebenen nicht-funktionalen Anforderungen, Anforderungen, die
auch typisch für Kiosksoftware im Allgemeinen sind. Sie sind für die Zielsetzung dieser Arbeit also von 
größerer Bedeutung.\\
Beim Definieren der Anforderungen beschränkt sich diese Arbeit auf die Anforderungen an die Software.
Dabei stellen manche Anforderungen auch Ansprüche an andere Teile des Systems. Diese Überschneidung
lässt sich nicht vermeiden, denn gerade die funktionalen Anforderungen stellen meist auch Anforderungen an das 
Gesamtsystem.\\ 

\section{Funktionale Anforderungen}
\label{funktionale}
\section{Nicht-funktionale Anforderungen}
\label{section:nicht-funktionale}