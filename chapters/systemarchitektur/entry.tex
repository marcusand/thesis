\chapter{Systemarchitektur}
\label{chap:systemarchitektur}

\begin{figure}
    \centering
    \includestandalone[width=1\textwidth]{figures/plant/out/ss-deployment-diagram}
    \caption{Deployment Diagram der \shst{}}
    \label{fig:ss-deployment-diagram}
\end{figure}

Für die weitere Betrachtung der Softwaretechnologien ist es erst nötig einen Überblick über 
das Gesamtsystem der \shst{} zu erhalten, um zu verstehen, wie die einzelnen Komponenten
zusammenarbeiten.\\
Dafür zeigt das Diagram in \autoref{fig:ss-deployment-diagram} die Topologie des Gesamtsystems.
Es soll dabei Überblick über alle im System enthaltenen Hardwarekomponenten und Applikationen
geben. Zusätzlich werden die Netzstruktur und die Kommunikationswege zwischen den Komponenten 
aufgezeigt.\\

Bei dem Diagram handelt es sich um ein UML Deployment Diagram \cite{uml-spec}. Es ist dafür 
gedacht, um die Systemarchitektur und die Verteilung von Hardware- und Softwarekomponenten in der 
Umgebung des Systems zu visualisieren \cite{uml-2.5}. Hardwarekomponenten werden dabei als Geräteknoten
(\emph{device}) dargestellt. Auf ihnen können Softwarekomponenten verteilt werden (\emph{service,
application, database}) \cite{uml-diagrams, uml-2.5}. Die Verbindungen zwischen Knoten und Komponenten
sind eine spezielle Art der Assoziation und stellen in diesem Fall Kommunikationspfade dar \cite{uml-2.5}.
Die Beschriftung an den Wegen nennt dabei das genutzte Kommunikationsprotokoll.\\
Die umschließenden Rahmen zeigen welche Komponenten sich in welchem Netz befinden.\\
Im Mittelpunkt der Architektur steht die Kiosk-Workstation -- der Ausstellungsrechner. Auf ihm läuft
die Clientapplikation (\emph{Sharing Station App}). An die Workstation direkt angeschlossen, und mit der
Applikation über die USB-Schnittstelle kommunizierend, ist die Webcam, 
welche für die Foto-Commitment Funktion (\ref{fa4}) benötigt wird. 
Die zwei weiteren Hardwarekomponenten \emph{Card Terminal} und \emph{Coin Slot} sind nicht 
an die Kiosk-Workstation, sondern über serielle Kommunikation an einen
Mikrocontroller (\emph{Embedded-Client}) angeschlossen. Dieser kommuniziert über
den \emph{MQTT Broker} mit der Clientapplikation. Die Hintergründe zu dieser Architektur werden in
\autoref{chap:hardwareschnittstellen} erläutert.\\
Alle eben beschriebenen Komponenten befinden sich im lokalen Netzwerk des Museums. Darüber hinaus gibt 
es noch die Dienste, welche über das Internet erreichbar sind. Im Zentrum steht hier der Server. 
Bei diesem handelt es sich um einen Cloud-Server mit Root-Zugriff. Auf diesem ist das CMS zusammen mit
der Datenbank gehostet. So ist das CMS über eine Weboberfläche im Internet erreichbar und erfüllt
damit die Anforderung \ref{nfa9}. Neben dem CMS und der Datenbank ist auf dem Server noch ein
Webserver gehostet, welcher die Applikationsfiles der \emph{Sharing
Station App} ausliefert. Die Hintergründe hierzu werden in \autoref{subs:deployment} beschrieben.
CMS sowie Webserver kommunizieren über das HTTPS-Protokoll mit der Clientapplikation.\\
Und schließlich gibt es noch den Wordpress-Server des Dialogmuseums. Dessen Architektur ist nicht bekannt,
aber auch nicht weiter von Bedeutung. Lediglich die Schnittstelle zum Eintragen einer E-Mail Adresse in 
die Newsletterliste ist relevant. Bei dieser handelt es sich um einen einzelnen Endpoint, welcher ebenfalls
über das HTTPS-Protokoll angesprochen werden kann. 