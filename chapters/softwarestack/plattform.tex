\subsection{Plattform}
\label{subs:plattform}

Nach dem Anforderungskatalog aus \autoref{chap:anforderungen}, betreffen die Anforderungen
\ref{nfa4} und \ref{nfa6} die Plattform der Clientapplikation.\\

\ref{nfa6} fordert die Plattformunabhängigkeit. Dies meint, dass die entwickelte
Applikation später, wenn gewünscht, in unterschiedlichen Umgebungen genutzt werden kann.
Also beispielsweise neben auf einer herkömmlichen Windows-Workstation auch auf einem
Tablet oder einer Linux-Workstation. Webtechnologien bieten hierfür die beste Vorraussetzung. 
Browser oder Umgebungen die Webanwendungen ausführen können, gibt es für praktisch 
jede Plattform. Wichtig ist jedoch trotzdem, dass bei der Entwicklung auf
Plattformunabhängigkeit geachtet wird. Beispielsweise bietet das Framework Electron.js die 
Möglichkeit, mit Webtechnologien native Desktop-Apps zu entwickeln. Es wird dabei aber nicht
nur mit Browser-Technologien gearbeitet -- ein parallel laufender Node.js-Prozess bietet
die Möglichkeit auf Betriebssystem-Funktionalitäten zuzugreifen 
und über einen Inter-Process-Communication (IPC) Kanal mit der Oberfläche der Anwendung zu
kommunizieren. Dies bietet zwar allerhand Vorteile, beispielsweise bei der Kommunikation
mit Hardwarekomponenten, bindet die Anwendung aber an eine Desktop-Umgebung. 
Bei der Entwicklung der Applikation wird im besten Fall also darauf geachtet, dass nur mit 
Browser-Technologien gearbeitet wird, da eine Bindung an eine spezifische Plattform so 
nicht stattfindet. Bei einem solchen Vorgehen wäre es später immer noch möglich, 
die Anwendung in einem Electron.js-Framework zu verpacken und als Desktop-Anwendung auszuliefern. 
Genauso könnte die Applikation aber auch als klassische Webanwendung im Internet verfügbar gemacht oder 
als mobile App, mit ein Framework wie Apache Cordova \cite{cordova}, ausgeliefert werden.\\

\ref{nfa4} fordert die Geschlossenheit der Clientapplikation. Im Folgenden wird davon ausgegangen, 
dass die Software auf einer Workstation und nicht beispielsweise auf einem mobilen Endgerät präsentiert
werden soll -- dies würde eine andere Herangehensweise erfordern.\\
Für die Geschlossenheit ist es im ersten Schritt zwingend notwendig, dass die Applikation in einem 
Vollbildmodus gestartet wird. Darüberhinaus sollten jegliche Betriebssystemfunktionen, die das 
Verlassen der Applikation möglich machen würden, deaktiviert werden. Electron.js bietet dafür beispielsweise
die \emph{kiosk} Option. Diese sorgt unter Anderem dafür, dass die Applikation im Vollbildmodus gestartet, 
die Menüleiste ausgeblendet und das Verlassen des Vollbildmodus deaktiviert wird.\\
Da sich jedoch zuvor entschieden wurde, eine reine Browser-Anwendung zu implementieren, ist es nun naheliegend 
diese auch in einem Browser und nicht als native Applikation zu präsentieren. Der Browser Google Chrome bietet
dafür ebenfalls eine \emph{kiosk} Option. Diese kann durch übergeben des Startparameters \texttt{-{}-kiosk}
aktiviert werden. Zusätzlich kann eine URL angegeben werden, die beim Öffnen direkt aufgerufen werden soll.
Der Browser startet dann ebenfalls in einem nicht verlassbaren Vollbildmodus und blendet jegliche Menüleisten aus.
Zusätzlich wird bei Touch-Input der Mauszeiger vollständig ausgeblendet.\\

\begin{figure}
  \lstinputlisting[%
    style=ES6, 
    caption=
  ]{webpack.prod.config.js}
  \caption{Teil der Webpack Konfiguration der \shst{} App}
  \label{lst:webpack-config}
\end{figure}
