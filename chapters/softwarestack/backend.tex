\section{Backend}
\label{sec:backend}

Für das Backend der \shst{} wurde auf ein bereits vorhandenes, von \meso{} entwickeltes
CMS zurückgegriffen. Es wurde jedoch vom Autor dieser Arbeit
an manchen Stellen angepasst und um einige Funktionen
erweitert. Es ist in Node.js \cite{node} geschrieben und speichert die Daten in einer 
MongoDB \cite{mongo}.\\
Da der Fokus dieser Arbeit nicht auf der Backend-Entwicklung liegen sollte, war es 
naheliegend hier ein bereits vorhandenes System zu nutzen. Mit das wichtigste Kriterium
bei der Wahl dieses Systems war die \emph{headless} Eigenschaft. Das CMS bildet also nur 
die Contentschicht und besitzt keine Präsentationsschicht. Der angelegte und verwaltete
Content wird der Clientapplikation über eine JSON-API zur Verfügung gestellt. 
Diese lose Kopplung zwischen Content- und Präsentationsschicht ist typisch für moderne
Webanwendungen und trägt in diesem Fall zur Erfüllung der Anforderung \ref{nfa8} bei,
welche fordert das System möglichst modular zu halten. Durch diese nicht monolithische 
Architektur ließe sich das CMS mit wenig Aufwand gegen ein anderes austauschen. Ebenso
könnten weitere Clientapplikationen entworfen werden, welche auf den gleichen Content
zugreifen.\\
Neben der Schnittstelle um die Daten auszuliefern, besitzt das CMS noch eine
Upload-Schnittstelle sowie eine Schnittstelle um Formulardaten abzuspeichern.\\

Das CMS ist Template-basiert \todo{Schreibweise checken}. Administratoren können Templates 
anlegen, welche eine Datenstruktur definieren. Redakteure können dann anhand dieser Templates 
konkrete Content-Objekte erstellen, welche baumartig miteinander verknüpft werden. Templates 
werden dabei mithilfe von verschiedenen Datenfeldern gebildet. Datenfelder sind zum Beispiel
Zahlenfeld, Textfeld oder Rich-Text-Feld. Datenfelder können dabei als übersetzbare Felder angelegt
werden. Beim Erstellen eines Content-Objekts können später so verschiedene Sprachversionen
hinterlegt werden, was eine wichtige Eigenschaft zur Erfüllung der Anforderung \ref{nfa3}
darstellt.\\

\todo{Screenshots vom CMS einfügen}
\todo{Links in RTF erläutern}


\iffalse
Template-basiert
Admins können Templates mit Datenstruktur erstellen
Redakteure können Objekte anhand von Templates erstellen
Sharing Station gibt Templates für Menü oder Modul
Navigationsstruktur der Oberfläche ist Baumartig - so sind auch die Daten
\fi