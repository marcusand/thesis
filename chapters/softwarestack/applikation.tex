\subsection{Applikation}
\label{subs:applikation}

Anforderungen aus \autoref{chap:anforderungen}, welche die eigentliche Applikation 
der Clientsoftware betreffen, sind die Anforderungen \ref{nfa2}, \ref{nfa3} und \ref{nfa8}.\\

\ref{nfa8} stellt die Anforderung der Modularität an das gesamte Softwaresystem. Diese
betrifft also auch die Clientapplikation. Wie in \autoref{sec:backend} bereits erläutert, 
ist für eine große Modularität eine lose Kopplung von Daten- und Präsentationsschicht ein
wichtiger Faktor. Umgekehrt bedeutet dies, dass die Clientsoftware eine in sich geschlossene
eigene Anwendung sein muss. Sie besitzt lediglich die Datenschnittstelle zum CMS. Das leistet im 
Falle der Webtechnologien eine Single-Page Applikation (SPA). Eine SPA wird einmalig von einem 
Server geladen. Laden von weiteren Seiten beim Navigieren durch die Oberfläche 
findet nicht statt \cite{js-definitive}. Um Transaktionen und neue Zustände möglich zu machen,
können jedoch im Hintergrund asynchrone Datenanfragen gemacht werden (AJAX). 
Die SPA gibt so den Nutzer*innen nicht mehr das Gefühl einer
Webseite, sonder einer geschlossenen Anwendung, was im Falle der Kiosksoftware ein gewünschter
Effekt ist und damit in gewisser Weise auch zur Erfüllung der Anforderung \ref{nfa4} beiträgt.\\
Das Laden einer einzigen Seite vom Server bedeutet jedoch nicht, dass keine Navigationsstruktur
mit Vor- und Zurück-Bewegungen implementieren werden kann. Die Navigation findet jedoch alleinig in 
der Clientapplikation statt und ist nicht mit Laden von neuen Seiten verbunden. Es wird lediglich
ein neuer Zustand der Applikation geladen \cite{spa-manifesto}.\\

\ref{nfa2} fordert Offline-Verfügbarkeit. Auch hier trägt das Konzept der SPA zur 
Erbringung der Anforderung bei. Dadurch, dass die Anwendung nur einmal initial geladen werden muss 
und danach vollständig im Client zur Verfügung steht, ist eine Netzwerkverbindung nach dem initialen 
Laden nicht mehr nötig. Das umfasst allerdings nur die eigentliche Applikation und nicht die
AJAX-Anfragen, die auch zu späteren Zeitpunkten erfolgen können. Auch ist bei einem Neuladen
der Applikation immer eine Netzwerkverbindung zwingend nötig. Das Konzept der SPA erfüllt
die Anforderung der Offline-Verfügung also nur zu einem gewissen Teil. Allerdings bildet die Geschlossenheit
einer SPA eine wichtige Voraussetzung um die Applikation auch vollständig offline zur Verfügung
zu stellen. Dies wird in \autoref{subs:plattform} weiter erläutert.\\

\ref{nfa8} fordert schließlich die Multilingualität der Oberfläche. Wie in \autoref{sec:backend} 
bereits erläutert, können Datensätze im CMS mehrsprachig angelegt werden. Diese werden der Clientapplikation
über eine Schnittstelle zur Verfügung gestellt und sind so zu jedem Zeitpunkt verfügbar. 
Nutzer*innen sollen später in der Oberfläche der Clientapplikation die Möglichkeit haben die Sprache
umzustellen. Erfolg dies, muss keine neue Seite vom Server angefordert werden, sondern lediglich
die angezeigten Daten durch die entsprechende Sprachversion ausgetauscht werden.\\

\begin{figure}
    \centering
    \includestandalone[width=1\textwidth]{figures/plant/out/ss-app-class-diagram}
    \caption{Komponenten Diagram der \shst{} App}
    \label{fig:ss-app-class-diagram}
\end{figure}

Um dieser beschriebenen Architektur gerecht zu werden, wurde für die \shst{} die JavaScript
Bibliothek React \cite{react} verwendet.
