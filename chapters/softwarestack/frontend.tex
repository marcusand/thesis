\section{Clientsoftware}
\label{sec:frontend}

\iffalse
- viele Anforderungen
- Fokus der Thesis
- beschreibt alle Schritte um eine Webandwendung in eine gelunge Kioskapplikation zu bringen
- Was meint Frontend: die eigentliche Kioskapplikation, Plattform: die Umgebung der Applikation, Deployment der Applikation
- SPA Routing im Frontend
- rein mit Webtechnologien, und noch weiter rein mit Browsertechnologien: beschränken auf HTML, CSS und JS
- mit Electron bindet man sich an die Plattform
\fi

Im Folgenden wird die Clientsoftware des Kiosksystems betrachtet. Damit ist der 
Softwareteil des Systems gemeint, welcher auf der Kiosk-Workstation vor 
Ort läuft. Diese wird in diesem Kapitel zur weiteren Betrachtung 
weiter unterteilt. Zum einen in die \emph{Applikation}, 
also die eigentliche den Nutzer*innen zugängliche Kioskapplikation.
Weiter wird die \emph{Plattform} der Clientsoftware betrachtet.
Plattform meint damit die Umgebung in der die Applikation läuft. Diese kann 
im Falle der Webtechnologien sehr unterschiedlich aussehen. 
Und schließlich wird das \emph{Deployment} besprochen -- also die Art wie die Applikation und 
Updates dieser auf die Kiosk-Workstation gelangen.\\

\subsection{Applikation}
\label{subs:applikation}

Anforderungen aus \autoref{chap:anforderungen}, welche die eigentliche Applikation 
der Clientsoftware betreffen, sind die Anforderungen \ref{nfa2}, \ref{nfa3} und \ref{nfa8}.\\

\ref{nfa8} stellt die Anforderung der Modularität an das gesamte Softwaresystem. Diese
betrifft also auch die Clientapplikation. Wie in \autoref{sec:backend} bereits erläutert, 
ist für eine große Modularität eine lose Kopplung von Daten- und Präsentationsschicht ein
wichtiger Faktor. Umgekehrt bedeutet dies, dass die Clientsoftware eine in sich geschlossene
eigene Anwendung sein muss. Sie besitzt lediglich die Datenschnittstelle zum CMS. Das leistet im 
Falle der Webtechnologien eine Single-Page Applikation (SPA). Eine SPA wird einmalig von einem 
Server geladen. Laden von weiteren Seiten beim Navigieren durch die Oberfläche 
findet nicht statt \cite{js-definitive}. Um Transaktionen und neue Zustände möglich zu machen,
können jedoch im Hintergrund asynchrone Datenanfragen gemacht werden (AJAX). 
Die SPA gibt so den Nutzer*innen nicht mehr das Gefühl einer
Webseite, sonder einer geschlossenen Anwendung, was im Falle der Kiosksoftware ein gewünschter
Effekt ist und damit in gewisser Weise auch zur Erfüllung der Anforderung \ref{nfa4} beiträgt.\\
Das Laden einer einzigen Seite vom Server bedeutet jedoch nicht, dass keine Navigationsstruktur
mit Vor- und Zurück-Bewegungen implementieren werden kann. Die Navigation findet jedoch alleinig in 
der Clientapplikation statt und ist nicht mit Laden von neuen Seiten verbunden. Es werden lediglich
neue Zustände geladen \cite{spa-manifesto}.\\

\ref{nfa2} fordert Offline-Verfügbarkeit. Auch hier trägt das Konzept der SPA zur 
Erbringung der Anforderung bei. Dadurch, dass die Anwendung nur einmal initial geladen werden muss 
und danach vollständig im Client zur Verfügung steht, ist eine Netzwerkverbindung nach dem initialen 
Laden nicht mehr nötig. Das umfasst allerdings nur die eigentliche Applikation und nicht die
AJAX-Anfragen, die auch zu späteren Zeitpunkten erfolgen können. Auch ist bei einem Neuladen
der Applikation immer eine Netzwerkverbindung zwingend nötig. Das Konzept der SPA erfüllt
die Anforderung der Offline-Verfügung also nur zu einem gewissen Teil. Allerdings bildet die Geschlossenheit
einer SPA eine wichtige Voraussetzung um die Applikation auch vollständig offline zur Verfügung
zu stellen. \emph{Vollständig offline} meint hier also, dass beispielsweise auch Daten, die über
einen AJAX-Aufruf aus dem CMS geladen werden, local gecached werden können. Zwar wären Content-Daten bei
Offlinestatus nicht mehr durch das CMS aktualisierbar, aber zumindest die letzte Version solange verfügbar,
bis ein Netzwerkzugriff wieder möglich ist. Um solch einer Anforderung gerecht zu werden, bieten moderne 
Browser die Service-Worker API \cite{service-worker-api}.\\ 
Bei Service-Workern handelt es sich um Proxy-Server, welche sich zwischen der Webanwendung und dem 
Netzwerk befinden. Ein Service-Worker wird in einem Skript definiert, welches
unabhängig vom Prozess der Webseite ausgeführt wird \cite{service-worker-intro}. 
Dieses speichert Netzwerkzugriffe in einem lokalen Speicher und kann Anfragen auf
diesen verweisen, sollte das Netzwerk bei dem nächsten Zugriff nicht erreichbar sein \cite{service-worker-api}.
Service-Worker werden bei initialem Aufruf heruntergeladen und bleiben auch nach Schließen des Browsers
gespeichert. Das gilt auch für die Daten, welche sie speichern. Diese sind also auch bei Neustart ohne
Netzwerkverbindung noch vorhanden und können angezeigt werden.\\
Eine vollständige Offline-Verfügbarkeit wäre somit fast erreicht. Einzig Transaktionen, bei welchen Eingaben
und Daten von Nutzer*innen gespeichert werden sollen, können noch nicht offline verfügbar gemacht werden.
Auch hier kommt die Technik der Service-Worker an ihre Grenzen \cite{service-worker-post}.\\
Dennoch gibt es auch hier Lösungen für dieses
Problem. Beispielsweise bietet die JavaScript Datenbank pouchDB \cite{pouchdb} die Möglichkeit POST-Anfragen
und Dateien in einer lokalen Datenbank solange zu speichern, bis eine Netzwerkverbindung wiederhergestellt ist,
um sie dann mit einer Backend-Datenbank zu synchronisieren. Dieser Ansatz wird jedoch im Rahmen dieser Arbeit,
aus Gründen des Umfangs, nicht weiter verfolgt.\\

\ref{nfa8} fordert schließlich die Multilingualität der Oberfläche. Wie in \autoref{sec:backend} 
bereits erläutert, können Datensätze im CMS mehrsprachig angelegt werden. Diese werden der Clientapplikation
in einem gemeinsamen Datensatz zur Verfügung gestellt und sind so zu jedem Zeitpunkt verfügbar. 
Nutzer*innen sollen später in der Oberfläche der Clientapplikation die Möglichkeit haben, die Sprache
umzustellen. Erfolgt dies, muss keine neue Seite oder neuer Datensatz vom Server angefordert werden, sondern lediglich
die angezeigten Daten durch die entsprechende Sprachversion ausgetauscht werden.\\

\begin{figure}
    \centering
    \includestandalone[width=1\textwidth]{figures/plant/out/ss-app-class-diagram}
    \caption{Komponenten Diagram der \shst{} App}
    \label{fig:ss-app-class-diagram}
\end{figure}

Um dieser beschriebenen Architektur gerecht zu werden, wurde für die Clientapplikation der \shst{} 
unter Anderen die JavaScript Bibliotheken React \cite{react} und Redux \cite{redux} zusammen mit dem 
Toolkit Workbox \cite{workbox} verwendet. React bildet dabei
den Rahmen für die SPA-Architektur; Workbox bietet eine Bibliothek, welche es erleichtert 
Service-Worker zu implementieren und Redux ist eine Bibliothek, welche ein Pattern für die Verwaltung 
von Applikationszuständen implementiert. Redux hilft bei steigender Komplexität von 
Zuständen in Anwendungen -- ist im Rahmen dieser Thesis aber von untergeordneter Bedeutung und wird daher
nicht weiter erläutert.\\

React bietet die Möglichkeit die meisten der zuvor genannten Anforderungen zu erfüllen. 
Das ist neben der Implementierung der SPA-Architektur, auch die Möglichkeit die 
Anwendung zu modularisieren. Denn: Neben der losen Kopplung von CMS und Clientapplikation
ist auch eine Modularität innerhalb der Clientapplikation nötig, um die Anforderung \ref{nfa8}
vollständig zu erfüllen. React erlaubt dies durch sein Komponenten-basiertes System. Einzelne
Interface-Elemente oder auch einzelne Unterseiten können als Komponenten gekapselt werden.
Wie fein diese Kapselung sein soll, kann von den Entwickler*innen entschieden werden. Richtig implementiert,
können Komponenten so wiederverwendet, ausgetauscht oder das System leicht um neue erweitert werden.\\
\autoref{fig:ss-app-class-diagram} zeigt die Komponentenstruktur der Clientapplikation der \shst{}.
Die Abbildung ist vereinfacht und zeigt nur die wichtigsten Komponenten. Die Darstellung orientiert
sich an einem UML-Klassendiagramm \cite{uml-spec}, jedoch sind die Klassen in diesem Falle keine Klassen sondern die
Komponenten 
\footnote{React-Komponenten können Klassen sein - müssen aber nicht. Seit React 16.8.0 und
der Einführung der Hook-API \cite{react-hooks} kann sogar gänzlich auf Klassen verzichtet werden und Komponenten 
durchgängig als Funktionen implementiert werden. Das wurde in diesem Falle so umgesetzt.}. 
Rot markierte Komponenten sind Container-Komponenten und haben keine eigene Darstellung, sondern implementieren 
lediglich Geschäftslogik. Blau markierte Komponenten sind Interface-Komponenten, die eine direkte 
Darstellung in der Oberfläche haben. Die einzelne gelb markierte Komponente ist der Redux-Store. 
Er ist ebenfalls eine besonderer Art der Container-Komponente und beinhaltet den Applikationszustand.\\
Im Zentrum der Abbildung befindet sich die Router-Komponente. Sie leitet zwischen den einzelnen Unterseiten
zu denen die Nutzer*innen navigieren können. Teilbäume, abgehend von der Router-Komponente, können als
die Hauptmodule der Applikation gesehen werden. Sie können dabei Untermenüs oder Module sein, welche
die konkreten Funktionalitäten wie die Spenden-Funktion (\ref{fa3}) oder die Newsletter-Funktion (\ref{fa5}) 
implementieren. Diese klarer Abgrenzung der einzelnen Module bietet so die Möglichkeit, diese leicht
durch andere zu ersetzten oder um neue zu erweitern.\\

\begin{figure}
  \lstinputlisting[%
    style=ES6, 
    caption=
  ]{webpack.prod.config.js}
  \caption{Ausschnitt der Webpack Konfiguration der \shst{} App}
  \label{fig:webpack-config}
\end{figure}

Workbox hilft bei der Implementierung von Service-Worker Skripten. Darüber hinaus gibt es ein 
Workbox-Plugin \cite{workbox-webpack-plugin} für den JavaScript-Bundler Webpack \cite{webpack}. 
Beide Technologien wurden im Falle der \shst{} eingesetzt. Mit Hilfe dieses Plugins müssen 
Service-Worker Skripte nicht selbst implementiert werden, sondern werden automatisiert erstellt.
Im Konfigurationsfile von Webpack kann durch setzen von Parametern bestimmt werden, was
der Service-Worker leisten soll. \autoref{fig:webpack-config} zeigt einen Teil der 
Webpack-Konfiguration der \shst{}. In Zeile 31 wird das Objekt zur Generierung des
Service-Workers erstellt. Als Parameter wird ein Objekt mit der Konfiguration übergeben. 
Diese Objekt ist optional -- auch ohne Konfiguration wird ein Service-Worker erstellt,
welcher dafür sorgt, dass alle von Webpack erstellten Files beim Nutzen der Applikation
gecached und offline verfügbar gemacht werden. Weitere Funktionalitäten können durch
Setzen der Konfigurationsparameter aktiviert werden.\\
In Zeile 34 wird die maximale Dateigröße gesetzt. 
Dateien größer als dieser Wert, werden nicht offline verfügbar gemacht. Dieser Wert
ist standardmäßig niedriger und wurde an dieser Stelle erhöht um auch das Video zu cachen, welches
für den Idle-Modus (\ref{fa8}) benötigt wird.

\begin{figure}
    \centering
    \includegraphics[width=1\textwidth]{figures/images/ss-network-falling-back-to-cache.png}
    \caption{NetworkFirst-Strategie. Quelle: The Offline Cookbook \cite{offline-cookbook}}
    \label{fig:network-first}
\end{figure}

In Zeile 35 wird definiert welche Dateien, neben den von Webpack erstellten Dateien, 
gecached werden sollen. Das sind in diesem Fall alle Daten und Dateien, die durch Netzwerkanfragen an 
das CMS geladen werden. Diese Anfragen haben den gemeinsamen Namespace \texttt{/api/} und 
können dadurch mit dem regulären Ausdruck \texttt{/.*\textbackslash/api\textbackslash/.*/} erfasst werden.
Mit der Option \texttt{handler} wird eine Caching-Strategie gewählt. Workbox bietet hier die Möglichkeit
zwischen fünf verschiedenen, vordefinierten Handler-Klassen zu wählen \cite{workbox-strategies}. 
Die Strategie der Klasse \texttt{NetworkFirst} beruht darauf, dass die Anfrage zuerst an das Netzwerk
gestellt wird. Ist dieses nicht erreichbar, wird auf die letzte Version im Cache zurückgegriffen. 
Ist ein Netzwerkzugriff erfolgreich, wird auch immer die Version im Cache durch diese Version ersetzt.
\autoref{fig:network-first} visualisiert diese Strategie. \todo{Strategie vielleicht nochmal überarbeiten}

\subsection{Plattform}
\label{subs:plattform}

Nach dem Anforderungskatalog aus \autoref{chap:anforderungen}, betreffen die Anforderungen
\ref{nfa4} und \ref{nfa6} die Plattform der Clientapplikation.\\

\ref{nfa6} fordert die Plattformunabhängigkeit. Dies meint, dass die entwickelte
Applikation später, wenn gewünscht, in unterschiedlichen Umgebungen genutzt werden kann.
Also beispielsweise neben auf einer herkömmlichen Windows-Workstation auch auf einem
Tablet oder einer Linux-Workstation. Webtechnologien bieten hierfür die beste Vorraussetzung. 
Browser oder Umgebungen die Webanwendungen ausführen können, gibt es für praktisch 
jede Plattform. Wichtig ist jedoch trotzdem, dass bei der Entwicklung auf
Plattformunabhängigkeit geachtet wird. Beispielsweise bietet das Framework Electron.js die 
Möglichkeit, mit Webtechnologien native Desktop-Apps zu entwickeln. Es wird dabei aber nicht
nur mit Browsertechnologien gearbeitet -- ein parallel laufender Node.js-Prozess bietet
die Möglichkeit auf Betriebssystem-Funktionalitäten zuzugreifen 
und über einen Inter-Process-Communication (IPC) Kanal mit der Oberfläche der Anwendung zu
kommunizieren \cite{electron-architecture}. Dies bietet zwar allerhand Vorteile, 
beispielsweise bei der Kommunikation mit Hardwarekomponenten, bindet die Anwendung aber an eine Desktop-Umgebung. 
Bei der Entwicklung der Applikation wird im besten Fall also darauf geachtet, dass nur mit 
Browsertechnologien gearbeitet wird, da eine Bindung an eine spezifische Plattform so 
nicht stattfindet. Bei einem solchen Vorgehen wäre es später immer noch möglich, 
die Anwendung in einem Electron.js-Framework zu verpacken und als Desktop-Anwendung auszuliefern. 
Genauso könnte die Applikation aber auch als klassische Webanwendung im Internet verfügbar gemacht oder 
als mobile App, mit ein Framework wie Apache Cordova \cite{cordova}, ausgeliefert werden.\\

\ref{nfa4} fordert die Geschlossenheit der Clientapplikation. Im Folgenden wird davon ausgegangen, 
dass die Software auf einer Workstation und nicht beispielsweise auf einem mobilen Endgerät präsentiert
werden soll -- dies würde eine andere Herangehensweise erfordern.\\
Für die Geschlossenheit ist es im ersten Schritt zwingend notwendig, dass die Applikation in einem 
Vollbildmodus gestartet wird. Darüberhinaus sollten jegliche Betriebssystemfunktionen, die das 
Verlassen der Applikation möglich machen würden, deaktiviert werden. Electron.js bietet dafür
die \emph{kiosk} Option. Diese sorgt unter Anderem dafür, dass die Applikation im Vollbildmodus gestartet, 
die Menüleiste ausgeblendet und das Verlassen des Vollbildmodus deaktiviert wird.\\
Da sich jedoch zuvor entschieden wurde eine reine Browseranwendung zu implementieren, ist es nun naheliegend 
diese auch in einem Browser und nicht als native Applikation zu präsentieren. Der Browser Google Chrome bietet
dafür ebenfalls eine \emph{kiosk} Option. Diese kann durch Übergeben des Startparameters \texttt{-{}-kiosk}
aktiviert werden. Zusätzlich kann eine URL angegeben werden, die beim Öffnen direkt aufgerufen werden soll.
Der Browser startet dann ebenfalls in einem nicht verlassbaren Vollbildmodus und blendet jegliche Menüleisten aus.
Zusätzlich wird bei Touch-Input der Mauszeiger vollständig ausgeblendet.\\

Für die Clientapplikation der \shst{} wurde, wie zuvor beschrieben, eine reine Browseranwendung implementiert,
um die Plattformunabhängigkeit zu gewährleisten. Für die Auslieferung der Software wurde auf ein Framework wie
Electron.js gänzlich verzichtet. Stattdessen wurde sich entschieden, durch Definieren eines so genannten Webmanifests,
eine installierbare Progressive Web App (PWA) zu erstellen \cite{web-app-manifest}. Dies bedeutet, 
dass die Applikation zum einem in einem Browser aufgerufen, aber auch durch Installieren dem Desktop 
der Workstation hinzugefügt werden kann. Diese installierte Applikation entspricht nichts 
weiter als einem Chrome-Browser, welcher bei Öffnen die gewünschte Seite direkt,
bzw. durch den zuvor definierten Service-Worker die Applikationsfiles aus dem lokalen Speicher, aufruft. Auch
können der Applikation die selben Startparameter wie dem Chrome Browser übergeben werden. Das ist wichtig,
um hier ebenfalls die \texttt{-{}-kiosk} Option nutzen zu können.
Zusätzlich kann für die installierte Version der Applikation ein Desktop-Icon definiert werden \cite{web-app-manifest}.
Dieses trägt dazu bei, den Eindruck einer nativen Applikation zu erwecken.\\
Wie schon beim Erstellen der Service-Worker, wurde auch für das Erstellen des Webmanifests auf ein 
Webpack-Plugin \cite{webpack-pwa-manifest-plugin} zurückgegriffen. Zeile 15-30 in \autoref{fig:webpack-config} 
zeigt die Einbindung dieser Plugins, welches während dem Build-Vorgang, mit den übergebenen Parametern, das Manifest erstellt. 

\subsection{Deployment}
\label{subs:deployment}

Die einzige Anforderung, welche das Deployment direkt betrifft, ist die 
\ref{nfa7}. Diese besagt lediglich, dass dies möglichst einfach und von außerhalb möglich
sein soll.\\
Bei nativen Applikation wäre hier ein Updater denkbar. Also ein Service,
welcher regelmäßig die Verfügbarkeit neuer Versionen auf einem Server prüft und einen Download 
dieser anbietet. Electron.js bietet für diesen Fall einige Möglichkeiten, diesen Prozess 
möglichst simpel zu gestalten. Beispielsweise mit dem eigenen 
\texttt{autoUpdater}-Modul \cite{electron-autoUpdater} oder Plugins wie 
\emph{electron-builder} \cite{electron-builder}. Auch sind mit beiden Möglichkeiten
automatische Updates umsetzbar.\\
Im Falle einer Applikation, die einzig mit Browsertechnologien implementiert wurde, sind 
noch simplere Strategien denkbar. Hier ist ein normales Webhosting der Applikationsfiles 
bereits eine ausreichende Deployment-Strategie. Wurde keine Offline-Verfügbarkeit durch
Service-Worker implementiert, können die statischen Files einer Browseranwendung
auch einfach auf dem Filesystem der Workstation hinterlegt und in einem Browser aufgerufen 
werden. Eventuelle Updates müssten in diesem Falle aber vor Ort oder durch einen
Fernzugriff auf den Rechner eingespielt werden. Im Fall des Webhostings genügt das Überspielen
der neuen Files auf einen Webserver.\\

Für die \shst{} wurde auf dem Server neben dem CMS auch ein Webserver installiert, welcher die 
Applikationsfiles ausliefert (\autoref{fig:ss-deployment-diagram}). Zusätzlich wurde über den 
firmeninternen GitLab-Server \cite{gitlab} eine Build- und Deployment-Pipeline eingerichtet, welche die 
Applikationsfiles automatisch generiert und vom Master-Branch des Repositorys auf den Webserver überträgt. 
Für ein Updaten der Clientapplikation, auf der Workstation in der Ausstellung, ist also nichts weiter 
nötig als die Änderungen auf den Master-Branch des Repositorys zu übertragen und die Applikation vor
Ort neu zu laden.\\
Das Webhosting der Applikation bietet außerdem den Vorteil, dass diese jeder Zeit in einem Browser 
über das Internet aufgerufen werden kann. Beim Anlegen von Daten im CMS kann dies so als
Vorschau genutzt werden. \todo{Screenshot der Browser Preview einfügen}
