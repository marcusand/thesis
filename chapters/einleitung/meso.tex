\section{MESO}
\label{sec:meso}

\mesoFull{} ist eine Digitalagentur mit Sitz in Frankfurt am Main.
Gegründet wurde sie 1997 und begann damals als Bürogemeinschaft von Programmierern
und Designern. Von Beginn an lag der Fokus auf Grafik, Musik, Elektronik, 3D-Visualisierungen
und Softwareentwicklung. Kommerzielle Projekte und eigene Projekte wurden gemeinschaftlich
umgesetzt.\\

Über die Jahre entstanden aus dieser Bürogemeinschaft mehrer Firmen: Aspekt1, \meso{} Web Scapes, \meso{}
Digital Interiors, \meso{} Image Spaces und \meso{} Digital Services. Bis zuletzt arbeiteten 
\meso{} Digital Interiors, \meso{} Digital Services und \meso{} Image Spaces als Firmenkollektiv zusammen, 
ehe sie im Jahr 2018 schließlich durch die drei Geschäftsführer zu der \mesoFull{} zusammengelegt
wurden.\\
Heute hat \meso{} rund 30 feste und einige freie Mitarbeiter*innen. Diese arbeiten zum Großteil vor Ort, teilweise
aber auch remote. Viele Mitarbeiter sind Designer oder Informatiker. Einige haben einen handwerklichen
Hintergrund, wie Schreiner oder Goldschmied. Geführt wird das Unternehmen von Sebastian Oschatz, Max Wolf
und Mathias Wollin.\\
So Transdisziplinär wie die Mitarbeiter sind auch die Projekte. \meso{} konzipiert, gestaltet und entwickelt
Ausstellungen, Messeauftritte, Showrooms, Webapplikationen und Apps. Die Projekte bewegen sich dabei meist an der
Schnittstelle zwischen Raum, Kommunikation und Technik. Kunden sind dabei oft Firmen aus der
Automobilindustrie (BMW, Merceds-Benz, Yanfeng oder Moovel), aus der Technologiebranche (HERE Technologies,
Siemens, Keyence) oder Institutionen öffentlicher Träger wie Museen, Hochschulen oder Städte (Hochschule Mainz,
Senckenberg Museum, Stadthalle Karlsruhe).\\

Bei Medieninstallationen und Echtzeit 3D-Grafik arbeitet \meso{} weitestgehend mit dem eigenentwickelten 
Tool VVVV \cite{vvvv} und mit der Spiel-Engine Unreal \cite{unreal}. Für Backend-Systeme, Apps, Interfaces
und Applikationen werden meist Webtechnologien eingesetzt.
