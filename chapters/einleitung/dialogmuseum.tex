\section{Dialogmuseum}
\label{sec:dialogmuseum}

Das Dialogmuseum Frankfurt existiert seit 2005 und ist ein privates, soziales 
Unternehmen. Es hat sich zum Ziel gesetzt informativ, integrativ und wirtschaftlich mit 
den Themen Blindheit und Sehbehinderung umzugehen. Dabei ist es ein Ort für sehende, 
blinde und sehbehinderte Menschen gleichermaßen \cite{dialogmuseum}. Allein im Jahr 2017 zählte das Museum 53.000 
Besuchende \cite{besucher}.\\

Das Ausstellungskonzept beruht auf dem 1988 von Prof. Dr. Andreas Heinecke entwickelten
\emph{Dialog im Dunkeln}. Dabei werden Besuchende von blinden Menschen durch eine 
völlig im Dunkeln gehaltene Ausstellung geführt. Dabei erleben sie Alltagssituationen
wie Szenen im Park, der Stadt oder einer Bar -- müssen sich dabei jedoch einzig auf
den Guide und ihre übrigen Sinne verlassen \cite{dialogmuseum}.\\
Das Ausstellungskonzept ist mittlerweile ein weltweites Franchisesystem, welches vom 
Dachverband Dialogue Social Enterprise GmbH (DSE) \cite{dachverband} mit Sitz in Hamburg verwaltet wird.
Zu Dialog im Dunkeln sind weitere Konzepte wie \emph{Dialog im Stillen} und Business Workshops
hinzugekommen. Diese schaffen weltweit Begegnungen integrativer Art und geben dabei 
Menschen mit Behinderung einen Arbeitsplatz.\\
\emph{Dialog im Dunkeln} zählt seit 1989 weltweit rund 6,5 Millionen Besucher und Besucherinnen
\cite{weltweit}.\\

Das Museum in Frankfurt ist seit Ende 2018 geschlossen und soll Anfang Mai 2020 in neuen Räumen, in der B-Ebene
der Haltestelle Frankfurter Hauptwache, wiedereröffnet werden. Während das Ausstellungskonzept
weitestgehend gleich bleibt, soll der Foyerbereich, sowie der Raum zu dem die Besucher zum Abschluss
der Führung gelangen, neu gestaltet werden. Dabei bekam \meso{} den Projektauftrag für zwei
digitale Exponate. Zum einen soll ein digitales, barrierefreies Gästebuch entstehen, sowie die
\shst{}, welche Gegenstand dieser Thesis ist. Das Gästebuch soll sich dabei später in dem 
genannten Reflexionsraum am Ende der Ausstellung und die \shst{} im Foyer des neuen Museums befinden.
