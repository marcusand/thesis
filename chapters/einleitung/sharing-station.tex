\section{\shst{}}
\label{sec:sharing-station}

Das im Rahmen dieser Thesis entwickelte Produkt trägt den Namen \shst{}. Der Name 
ist eine interne Bezeichnung und soll die zwei Haupteigenschaften des Produkts zusammenbringen.
\emph{Sharing}, da das Produkt in der Hauptaufgabe Informationen mit Benutzern teilen soll. 
Und \emph{Station} da es sich um ein Kiosksystem handelt.\\

Das Produkt besteht dabei aus der Software und mehreren Hardwarekomponenten. In der Hauptsache
wird sich diese Arbeit mit der Softwareentwicklung des Produkts beschäftigen. In \todo{Kapitel einfügen} wird zwar
auf die Hardwareanbindung eingegangen, jedoch liegt auch hier der Fokus auf der Softwarekommunikation.\\
Zu erwähnen sei auch, dass der Autor sich in der Entwicklung des Produkts hauptsächlich um die Softwareentwicklung 
gekümmert hat. Produkt-, Screen- und Hardwaredesign sowie das Projektmanagement wurden von anderen Mitarbeitern
der Firma \meso{} übernommen.\\

Ausgangspunkt für die Idee der \shst{} war der Umzug und die damit geplante Neueröffnung des Dialogmuseums.
Es entstand der Wunsch im Foyer des Museums den Besucher*innen die Möglichkeit zu geben sich zum einen 
über das Museum und den Dachverband, sowie über andere ähnliche Projekte zu informieren. Auch sollte
es die Möglichkeit geben sich aktiv zu engagieren und zu beteiligen, beispielsweise durch Spenden oder 
Hinterlassen seiner Kontaktdaten. Und weiter gab es den Wunsch, den Besucher*innen die Möglichkeit zu geben ein 
öffentliches, soziales Commitment zu hinterlassen.\\
Schon früh war die Idee da diese Anforderungen nicht beispielsweise durch Drucken und Auslegen von 
Broschüren und Flyern sowie Aufstellen einer Spendenbox zu erfüllen, sondern das Ganze in einer digitalen
Form anzubieten. So entstand die Idee der \shst{}: Eine Art Terminal mit großem Touchscreen, an dem die Besucher*innen 
all die zuvor beschriebenen Dinge tun können und welches durch seine digitale Form in seinem Funktionsumfang
beliebig erweiterbar ist.

\subsection{\shst{} als Produkt}
\label{subsection:sharing-station-produkt}

Von Anfang an wurde die Software der \shst{} unter den Gesichtspunkten der Adaptier- und Wiederverwendbarkeit
entwickelt. Schon früh wurde von \meso{} erkannt, dass die Anforderungen, welche das Dialogmuseum an die \shst{} hat, 
Anforderungen sind, die in der Praxis häufig von Museen und Ausstellungsmachern gestellt werden. Die Forderung
nach Alternativen zu klassischen Ausstellungsheften, Flyern und Newsletter-Listen, wird gerade im Zuge von
Digitalisierung im Museums- und Ausstellungsbetrieb sehr häufig gestellt.
So entstand die Idee, die \shst{} für das Dialogmuseum so zu entwickeln, dass sie später als Produkt auch
anderen Kunden angeboten werden kann.\\
Schon während der Entwicklungszeit gab es erstes Interesse innerhalb des Dialogue Social Enterprise GmbH Verbands, 
die \shst{} in anderen Museen einzusetzen. 