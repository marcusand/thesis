\section{Web APIs}
\label{sec:web-apis}

\begin{figure}
  \lstinputlisting[%
    style=ES6,
    caption=
  ]{webcam.js}
  \caption{Zugriff auf die Webcam über die \emph{Media Capture and Streams API}}
  \label{fig:webcam}
\end{figure}

Die Web APIs bilden Schnittstellen, mit denen eine Browserandwendung, typischerweise über ein JavaScript-Interface,
kommunizieren kann \cite{web-apis}. Manche APIs bieten dabei auch Zugriff auf ausgewählte 
Betriebssystemschnittstellen und Hardware. Beispielsweise kann mit der \emph{Web Audio API} auf 
angeschlossene Aufnahme- und Ausgabequellen für Audio zugegriffen werden \cite{web-audio-api}.\\
Für die in \ref{fa4} beschriebene
Foto-Commitment-Funktion wird das Bild einer angeschlossenen Webcam benötigt. Das ist mit dem \emph{Media Capture and 
Streams API} möglich \cite{media-stream}. \autoref{fig:webcam} zeigt den Zugriff über die API. In Zeile 16 wird 
über die \texttt{getUserMedia()} Funktion ein \texttt{MediaStream} Objekt angefordert. Über das Konfigurationsobjekt
kann genau spezifiziert werden, welche Art von Stream vom Betriebssystem angefordert werden soll. In diesem Fall
wird ein reiner Video-Stream mit einer 4k-Auflösung im Seitenverhältnis 1:1 gefordert. Der Zugriff muss 
im Browser über ein Dialogfenster bestätigt werden, anschließend kann das \texttt{MediaStream} Objekt als
Videoquelle in der Anwendung genutzt werden. Der \texttt{insertWebcamStream} Funktion in Zeile 23 kann dafür
ein HTML Video-Element als Parameter übergeben werden. In Zeile 27 wird das Stream-Objekt als Quelle
des Video-Elements gesetzt und anschließend in Zeile 30 gestartet.

\iffalse
- Web USB
- 
\fi