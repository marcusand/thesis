\section{Kommunikation über das MQTT/Websocket Protokoll}
\label{sec:mqtt-websocket}

Um nicht von den Möglichkeiten der Web APIs abhängig zu sein und die Kommunikation
zu weiteren Geräten, wie beispielsweise einem Kartenlesegerät und Münzschlitz (\ref{fa3}),
zu ermöglichen, wird im Rahmen dieser
Arbeit eine generische Lösung entwickelt, um die Kommunikation zwischen Hardwaregeräten
und einer Browserapplikation zu ermöglichen. Darüberhinaus wird der Grundstein 
einer Konvention für zukünftige Anwendungsfälle ähnlicher Art gelegt.\\

\subsection{Websocket}
Für eine sinnvolle Kommunikation zwischen einem Gerät und einer Anwendung ist meist
eine bidirektionale Verbindung nötig. Ein Protokoll, welches die bidirektionale Verbindung
bei Webanwendungen ermöglicht, ist das Websocket Protokoll \cite{rfc-websocket}. Über dieses
kann eine Browseranwendung zum einen Nachrichten an einer Server übermitteln, zum anderen 
aber auch zu jedem Zeitpunkt Nachrichten vom Server empfangen und auf diese reagieren.

\subsection{MQTT über Websocket}
\subsection{meso-connect}
