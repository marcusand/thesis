\chapter{Hardwareschnittstellen}
\label{chap:hardwareschnittstellen}

In diesem Kapitel wird die Kommunikation mit Hardwarekomponenten betrachet.
\ref{nfa1} fordert, dass die Clientsoftware zusammen mit Hardwarekomponenten funktionieren muss. Wie in
der Anforderung bereits erläutert, sind Hardwarekomponenten meist ein essenzieller Bestandteil von Kiosksystemen
und die Steuerung dieser, durch die Clientsoftware, muss gewährleistet werden. Im Falle der webbasierten 
Softwaretechnologien stößt man hier auf einige Hürden.\\
Während Serveranwendungen, native Anwendungen und auch der Node.js-Prozess einer 
Electron-Anwendung Zugriff auf eine Vielzahl von Betriebssystemschnittstellen haben \todo{Node API}
und somit leicht mit Hardwarekomponenten kommunizieren können, sind
die Möglichkeiten einer reinen Browseranwendung wesentlich limitierter. Browseranwendungen und Webseiten wird
typischerweise, aus sicherheitstechnischen Gründen, kein direkter Zugriff auf Betriebssystem-APIs gegeben.
Nur einige ausgewählte Funktionalitäten werden durch die Web APIs zur Verfügung gestellt \cite{web-apis}.
Für Anwendungsfälle, die über die Möglichkeiten der Web APIs hinausgehen, müssen eigene Lösungen entworfen 
werden. Hierzu wurde im Rahmen dieser Arbeit eine Lösung über das MQTT/Websocket-Protokoll entwickelt.

\section{Web APIs}
\label{sec:web-apis}

\begin{figure}
  \lstinputlisting[%
    style=ES6,
    caption=
  ]{webcam.js}
  \caption{Zugriff auf die Webcam über die \emph{Media Capture and Streams API}}
  \label{fig:webcam}
\end{figure}

Die Web APIs bilden Schnittstellen, mit denen eine Browserandwendung, typischerweise über ein JavaScript-Interface,
kommunizieren kann \cite{web-apis}. Manche APIs bieten dabei auch Zugriff auf ausgewählte 
Betriebssystemschnittstellen und Hardware. Beispielsweise kann mit der \emph{Web Audio API} auf 
angeschlossene Aufnahme- und Ausgabequellen für Audio zugegriffen werden \cite{web-audio-api}.\\
Für die in \ref{fa4} beschriebene
Foto-Commitment-Funktion wird das Bild einer angeschlossenen Webcam benötigt. Das ist mit dem \emph{Media Capture and 
Streams API} möglich \cite{media-stream}. \autoref{fig:webcam} zeigt den Zugriff über die API. In Zeile 16 wird 
über die \texttt{getUserMedia()} Funktion ein \texttt{MediaStream} Objekt angefordert. Über das Konfigurationsobjekt
kann genau spezifiziert werden, welche Art von Stream vom Betriebssystem angefordert werden soll. In diesem Fall
wird ein reiner Video-Stream mit einer 4k-Auflösung im Seitenverhältnis 1:1 gefordert. Der Zugriff muss 
im Browser über ein Dialogfenster bestätigt werden, anschließend kann das \texttt{MediaStream} Objekt als
Videoquelle in der Anwendung genutzt werden. Der \texttt{insertWebcamStream} Funktion in Zeile 23 kann dafür
ein HTML Video-Element als Parameter übergeben werden. In Zeile 27 wird das Stream-Objekt als Quelle
des Video-Elements gesetzt und anschließend in Zeile 30 gestartet.

\iffalse
- Web USB
- 
\fi
\subimport{meso-connect/}{entry}