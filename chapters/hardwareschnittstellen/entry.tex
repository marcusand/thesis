\chapter{Hardwareschnittstellen}
\label{chap:hardwareschnittstellen}

In diesem Kapitel wird die Kommunikation mit Hardwarekomponenten genau betrachet.
\ref{nfa1} fordert, dass die Clientsoftware zusammen mit Hardwarekomponenten funktionieren muss.\\
Während Serveranwendungen und der Node-Prozess einer Electron.js-Anwendung Zugriff auf die meisten
Betriebssystemschnittstellen haben und somit leicht mit Hardwarekomponenten kommunizieren können, sind
die Möglichkeiten einer reinen Browseranwendung wesentlich limitierter. Browseranwendungen werden
typischerweise, aus sicherheitstechnischen Gründen, kein direkten Zugriff auf Betriebssystem-APIs gegeben.
Nur einige ausgewählte Funktionalitäten werden durch die Web APIs zur Verfügung gestellt \cite{web-apis}.
Für Anwendungsfälle, die über die Möglichkeiten der Web APIs hinausgehen, müssen eigene Lösungen entwickelt 
werden. Hierzu wurden im Rahmen dieser Arbeit eine Lösung über das MQTT/Websocket Protokoll entwickelt.

\section{Web APIs}
Die Web APIs bilden Schnittstellen, mit denen eine Webanwendung, unter Anderem, mit Betriebssystemschnittstellen
kommunizieren können. Mit der Web Audio API kann beispielsweise auf die angeschlossenen Aufnahme- und Ausgabequellen
für Audio zugegriffen werden. 
\section{Websocket-basierte Kommunikation}