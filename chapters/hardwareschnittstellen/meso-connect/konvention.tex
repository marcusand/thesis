\subsection{Konvention}
\label{subs:konvention}

Basierend auf der Kommunikation über das MQTT/Websocket Protokoll und der Homie Konvention,
wurde im Rahmen dieser Arbeit eine eigene Konvention entwickelt. Diese soll die Kommunikation
zwischen Hardwaregeräten und Browseranwendungen standardisieren. Der Einsatz beschränkt
sich dabei nicht nur auf Kiosksysteme, sondern ist auch in anderen Szenarien denkbar.\\
Im Folgenden werden die wichtigsten Eigenschaften der Konvention erläutert -- die vollständige
Konvention befindet sich in \autoref{app:meso-connect}. Auch sei an dieser Stelle erwähnt,
dass die Konvention in Absprache und teilweise zusammen mit anderen Mitarbeitern 
der Firma \meso{} entwickelt wurde.\\

Die Konvention setzt dabei im ersten Schritt die Kommunikation über das MQTT Protokoll voraus.
Um Hardwaregeräte über das MQTT Protokoll ansprechen zu können sind unterschiedliche Lösungen 
denkbar. Diese sind vom jeweiligen Gerät und Anwendungsfall abhängig. Bei einfachen Sensoren oder
Geräten mit serieller Schnittstelle ist der Anschluss an einen Mikrocontroller denkbar.
Dieser muss entsprechend programmiert werden, um den Zugriff auf die angeschlossenen Geräte 
oder Sensoren über eine MQTT Schnittstelle zu ermöglichen.\\
Um eine Browseranwendung in die Kommunikation einzubinden, wird die Kommunikation über das MQTT/Websocket
Protokoll vorausgesetzt. Der Broker muss dafür die Kommunikation auf zwei verschiedenen 
Netzwerkports ermöglichen. Zum Beispiel Port 1883 für reguläre MQTT-Nachrichten und Port 9001
für Nachrichten über das Websocket/MQTT Protokoll.\\
Nach diesem Setup ist eine bidirektionale, flexible Datenverbindung zwischen Browseranwendung
und Broker hergestellt.\\

Um die Kommunikation zu vereinheitlichen, werden einige Grundsätze der Homie Konvention übernommen.
Das ist zum ersten die Unterscheidung in Devices, Nodes und Properties und die sich daraus 
ergebende Topic-Struktur. Weiter wird das Typsystem für die Payload-Daten übernommen. 
Dieses wird allerdings um den \emph{JSON}-Typ erweitert. 
Ein JSON-Objekt könnte auch als String übertragen werden -- dann wäre allerdings
das Überprüfen der Wohlgeformtheit des JSON-Objekts nicht Teil der Konvention.
Da in der Praxis, und gerade bei Projekten der Firma \meso{}, oft mit JSON-Objekten in der 
MQTT-Kommunikation gearbeitet wird, ist das standardmäßige Prüfen von JSON-Objekten eine
sinnvolle Erweiterung.

\begin{figure}
  \lstinputlisting[%
    style=ES6,
    caption=
  ]{meso-connect-config-blueprint.json}
  \caption{meso-connect Konfigurationsfile}
  \label{fig:meso-connect-config-blueprint}
\end{figure}

Die Hauptsache in der sich meso-connect von Homie unterscheidet, ist die Konfiguration. meso-connect
sieht dafür ein File vor. Dieses, im JSON-Format gehaltene File, ist Grundlage jeder Kommunikation
mit der meso-connect Konvention. Jeder Client wird dabei mit dem selben File konfiguriert -- es muss 
deshalb nur einmalig erstellt werden und kann in einer Versionsverwaltung gehalten werden. 
\autoref{fig:meso-connect-config-blueprint} zeigt die Vorlage für ein solches Konfigurationsfile. 
Gut zu erkennen ist die Unterscheidung von Devices, Nodes und Properties. Diese sind jeweils Arrays --
es können also jeweils beliebig viele eingetragen werden. Wichtig ist, dass jeder MQTT-Client als ein
Device im Konfigurationsfile aufgelistet werden muss. Auch dann, wenn der Client selber 
keine Nodes und Properties besitzt. Das Konfigurationsfile gibt so immer einen Überblick über die 
gesamte MQTT Kommunikation eines Systems und kann für die Konfiguration jedes Clients genutzt werden.\\
Die im Konfigurationsfile angegebenen Attribute sind von der Homie Konvention übernommen. Einzig \emph{id} heißt
bei Homie \emph{name}. Diese Attribute werden bei Homie auf eigenen Topics veröffentlicht, um die 
Auto-Discovery Funktion möglich zu machen. Darauf verzichtet meso-connect. Da jeder Client das gemeinsame
Konfigurationsfile zur Grundlage hat, wäre das Veröffentlichen der Attribute redundant. Ein Auto-Discovery
Feature ist im Kontext von Kiosk- bzw. Installationssystemen für Museen und Ausstellungen ohnehin weniger wichtig.
Solche Systeme sind, mit all ihren Clients, meist im Voraus ausreichend bekannt. Das Konfigurationsfile
bietet dafür in diesem Kontext größere Vorteile, da es die Konfiguration der einzelnen Clients stark
vereinfacht \todo{Vorteile ausführen}. Die einzigen Attribute, welche auf Topics zur Laufzeit veröffentlicht 
werden, sind

\[\texttt{meso-connect/devicename/\$state} \]

und

\[\texttt{meso-connect/devicename/nodename/propertyname/\$error} \]

Das Attribut \texttt{\$state} ist aus der Homie Konvention übernommen und gibt Information über den aktuellen
Zustand des Geräts. Dieser Wert ist dynamisch und kann sich zur Laufzeit verändern. \texttt{\$error} ist nicht 
Teil der Homie Konvention. Aber auch dieser Wert kann nur dynamisch gesetzt werden.
Auf diesem Topic sollen eventuell auftretende Fehler in der Kommunikation veröffentlicht
werden. Das könnten zum Beispiel Fälle sein, in denen Daten verschickt werden, die nicht dem Typ entsprechen.
Oder wenn ein Client versucht ein Property-Wert zu ändern, welcher nicht setzbar ist. Beispielsweise können so,
durch abonnieren des Topics

\[\texttt{meso-connect/+/+/\$error} \]

alle auftretenden Fehler in einem System erfasst werden.\\

Die meso-connect Konvention wurde bei der \shst{} angewandt. Wie \autoref{fig:ss-deployment-diagram} zeigt,
sind dafür Kartenterminal und der Sensor des Münzschlitz an einem Mikrocontroller angeschlossen. Dieser
kommuniziert über Netzwerk und das MQTT Protokoll mit dem Broker. Bei dem Broker handelt es sich um 
eine Mosquitto-Instanz \cite{mosquitto}, welche auf der Kiosk-Workstation installiert ist. \todo{wirklich mosquitto?}
Der Broker wiederum kommuniziert über das Websocket/MQTT Protokoll
mit der Clientapplikation.\\
Für die eigentliche Kommunikation über das Protokoll wurde im Rahmen dieser Arbeit und für die Clientapplikation 
der \shst{} eine JavaScript Bibliothek implementiert, welche in einer Browser- oder Node.js-Anwendung
nutzbar ist. Diese implementiert die gesamte meso-connect Konvention und ist für den universellen Einsatz
konzipiert. Die vollständige API-Dokumentation dieser Bibliothek befindet sich in \autoref{app:meso-connect-js}.

\begin{figure}
  \lstinputlisting[%
    style=ES6, 
    caption=
  ]{data/sharing-station-app/meso-connect-config.json}
  \caption{meso-connect Konfigurationsfile der \shst{}}
  \label{fig:shst-meso-connect-config}
\end{figure}

\autoref{fig:shst-meso-connect-config} zeigt das bei der \shst{} genutzte Konfigurationsfile für die 
meso-connect Kommunikation.

\iffalse
- Konfigurationsfile, Beispiel der JS Bib 
- JS API Dokumentation in den Anhang
\fi