\subsection{Websocket und MQTT}
\label{subs:websocket-und-mqtt}

Für eine sinnvolle Kommunikation zwischen einem Gerät und einer Anwendung ist meist
eine bidirektionale Verbindung nötig. Ein Protokoll, welches bei Webanwendungen die 
bidirektionale Verbindung erlaubt, ist das Websocket Protokoll \cite{rfc-websocket}. Über dieses
kann eine Browseranwendung zum einen Nachrichten an einer Server übermitteln, zum anderen 
aber auch zu jedem Zeitpunkt Nachrichten vom Server empfangen und auf diese reagieren.
Im TCP/IP-Modell ist Websocket dabei ein Protokoll, welches auf der Anwendungsschicht arbeitet. 
Dennoch kann es selbst auch eine Transportschicht für weitere Protokolle sein \cite{websocket-definitive}.
Um in Browseranwendungen mit dem Websocket Protokoll zu arbeiten, stellen die meisten 
Browser die Websocket API zur Verfügung \cite{websocket-api}.\\

MQTT ist ein auf dem publish/subscribe-Pattern basierendes Nachrichten-Protokoll \cite{mqtt-standard}.
Es wird häufig im den Bereichen IoT und Home-Automation eingesetzt. Es folgt einer
klassischen Server-Client Architektur, wobei der Server in der MQTT Umgebung \emph{Broker}
genannt wird. Clients können über den Broker Nachrichten-Topics abonnieren oder Nachrichten
über den Broker auf Topics versenden. Der Broker leitet versendete Nachrichten an alle Clients
weiter, die das entsprechende Topic abonniert haben. Topics sind Namespaces, welche hierarchisch
aufgebaut sind. Das Slash-Zeichen wird dabei als Trennzeichen genutzt. Ein Topic kann also zum 
Beispiel \texttt{home/living-room/temperature} lauten. Mit Wildcards (+ und \#) können auch ganze
Namespaces, anstelle einzelner Topics, abonniert werden \cite{mqtt-man-page}.\\
Über MQTT können Nachrichten in beliebiger Encodierung oder als binäre Daten versendet werden
\cite{mqtt-essentials-part-4}.\\

Wie eingangs erwähnt, kann das Websocket Protokoll Transportschicht für
höhere Protokolle sein. So kann es auch die Transportschicht für MQTT sein und eine Browseranwendung
kann zum MQTT-Client werden. Da das Websocket Protokoll alle nötigen Eigenschaften besitzt, ist es
möglich MQTT Pakete über Websocket Pakete zu versenden. Die einzige Voraussetzung hierfür ist, dass der Broker
die Option der Kommunikation über Websockets unterstützt. Der beliebte Open Source Broker Mosquitto \cite{mosquitto}
besitzt diese Option.