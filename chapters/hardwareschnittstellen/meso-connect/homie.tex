\subsection{Homie}
\label{subs:homie}

Ein Designprinzip von MQTT lautet Einfachheit \cite{mqtt-design-principles}. 
Das Protokoll gibt nicht vor wie Topics aufgebaut werden und Daten benötigen keine spezielle
Encodierung und Typisierung. Eine Kommunikation muss also immer selbst und
für jedes Projekt neu konfiguriert werden. Das ist aufwendig und führt in der 
Praxis häufig zu Fehlern.\\

\begin{figure}
  \lstinputlisting[%
    style=ES6,
    caption=
  ]{homie-example.txt}
  \caption{Homie Topic Struktur. Quelle: \cite{homie}}
  \label{fig:homie-example}
\end{figure}

Homie \cite{homie} bietet dafür eine Konvention, welche diese Probleme lösen soll. Dabei ist Homie
ein Satz an Regeln, welcher in einigen Softwarebibliotheken, für verschiedene
Sprachen und Plattformen, implementiert wurde. Die Konvention schreibt dabei vor, wie Topics 
gebildet werden und führt eine Typisierung für die Payload-Daten ein.
Um Topics zu bilden, unterscheidet Homie zwischen Devices, Nodes und Properties. Ein Device
ist ein physisches Gerät, wie beispielsweise ein Mikrocontroller.
Ein Device kann mehrere Funktionen besitzen -- diese werden als Nodes
bezeichnet. Nodes wiederum können mehrere Properties haben. Properties sind Eigenschaften, welche 
eine konkreten Wert besitzen können oder deren Wert von anderen Clients gesetzt werden kann.
Aus dieser Unterscheidung heraus werden die Topics nach der Struktur
\[\texttt{homie/device/node/property} \]
gebildet. \autoref{fig:homie-example} zeigt ein vollständiges Beispiel. Dort ist auch zu sehen, dass 
Metadaten zu Devices, Nodes und Properties über spezielle Topics, mit dem Prefix \texttt{\$},
veröffentlich werden. Zu den Metadaten auf Property-Ebene zählt unter Anderem die Angabe eines
Datentyps. Homie gibt dafür die Typen String, Integer, Float, Boolean, Enum und Color vor.
Weitere Metadaten sind das Attribut \texttt{\$settable}, mit dem bestimmt werden kann
ob ein anderer Client den Wert setzten darf. Oder das Attribut \texttt{\$retained}, über 
welches gesteuert werden kann, ob es sich bei dem Property um ein Event handelt oder einen Wert,
welcher dauerhaft auf dem Topic ausgelesen werden kann. Es existieren noch einige weitere Attribute,
welche aber aus Gründen des Umfangs nicht weiter erläutert werden.\\
Geräte, welche die Homie Konvention implementieren, sind praktisch konfigurationslos mit anderen
Homie Clients verwendbar. Denn: Durch die konsistente Veröffentlichung der Metadaten unter 
den speziellen Topics ergibt sich eine Autodiscovery-Eigenschaft. Clients können durch Abonnieren
von Wildcards andere Geräte finden und deren Eigenschaften erkennen. Zusätzlich ist durch die
konsistente Typisierung mit vorgegebenen Datentypen, eine standardisierte Kommunikation möglich. 

\iffalse
- Ein Design Prinzip ist Einfachheit (mqtt.github.io)
- MQTT ist ein sehr simples Protokoll
- Aufbau von Topics kann beliebig geschehen. MQTT gibt keine Struktur vor
- Payloads können jegliche Encodierung haben

- typisiert Daten
- Metadaten über spezielle Topics die mit \$ starten
- Autodiscovery

- Unterscheidet zwischen device node und property
- Beispiel von Webseite
\fi