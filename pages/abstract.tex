\chapter{\abstractname}

Diese Bacherloarbeit untersucht das Entwickeln von Kiosksoftware mit Webtechnologien
anhand eines realen Praxisbeispiels. Dabei werden im ersten Schritt die Anforderungen
an Kiosksoftware analysiert. Dies geschieht nach einer typischen Methode der Systemanalyse
und der Unterteilung in funktionale und nicht-funktionale Anforderungen. 
Anhand der nicht-funktionalen Anforderungen
wird der Softwarestack strukturiert betrachtet und Lösungen mit Hilfe von Patterns und 
Software-Bibliotheken entwickelt. 
Dabei zeigen sich große Vorteile von Webtechnologien bei den Themen
Plattformunabhängigkeit und Deployment-Strategien, während man bei Offline-Verfügbarkeit und der
Anbindung von Hardware auf Grenzen stößt. Dafür wird im letzten Kapitel das Problem der
Hardwareanbindung genau betrachtet und eine standardisierte Konvention auf
Basis des MQTT/Websocket Protokolls und der Homie Konvention entworfen.\\
Die Arbeit richtet sich dabei an Softwareentwickler -- ein Grundverständnis für typische 
Software- und Webtechnologien sollte gegeben sein. Nach Lesen der Arbeit soll klar sein
wie bei der Entwicklung von Kiosksoftware mit Webtechnologien vorgegangen werden muss, welche
Vor- und Nachteile sie bieten und wie ein typisches Kiosksystem mit Webtechnologien funktionieren
kann. \\

\iffalse
- wie typische Anforderungen an Kiosksoftware mit Webtechnologien erfüllt werden können
- zum Beispiel Offline Verfügbarkeit und Geschlosssenheit 
\fi

\todo{Englisches Abstract}
