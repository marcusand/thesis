\chapter{\abstractname}

Diese Bacherloarbeit untersucht das Entwickeln von Kiosksoftware mit Webtechnologien
anhand eines realen Praxisbeispiels. Dabei werden im ersten Schritt die Anforderungen
an Kiosksoftware analysiert. Dies geschieht nach einer typischen Methode der Systemanalyse
und der Unterteilung in funktionale und nicht-funktionale Anforderungen. 
Anhand der nicht-funktionalen Anforderungen
wird der Softwarestack strukturiert betrachtet und Lösungen mit Hilfe von Patterns und 
Software-Bibliotheken entwickelt. 
Dabei zeigen sich große Vorteile von Webtechnologien bei den Themen
Plattformunabhängigkeit und Deployment-Strategien, während man bei Offline-Verfügbarkeit und der
Anbindung von Hardware auf Grenzen stößt. Dafür wird im letzten Kapitel das Problem der
Hardwareanbindung genau betrachtet und eine standardisierte Konvention auf
Basis des MQTT/Websocket-Protokolls und der Homie Konvention entworfen.\\
Die Arbeit richtet sich dabei an Softwareentwickler -- ein Grundverständnis für typische 
Software- und Webtechnologien sollte gegeben sein. Nach Lesen der Arbeit soll klar sein
wie bei der Entwicklung von Kiosksoftware mit Webtechnologien vorgegangen werden muss, welche
Vor- und Nachteile sie bieten und wie ein typisches Kiosksystem mit Webtechnologien funktionieren
kann. \\

This bachelor thesis investigates the development of kiosk software with web technologies based
on a practice-proven example. At first, the requirements are analyzed based on a typical
method of requirements engineering and the classification in functional and non-functional requirements. 
Based on the non-functional requirements the software stack gets analyzed and solutions are 
developed by means of patterns and software libraries. Great advantages of web technologies
become apparent when it comes to cross-platform compatibility and deployment strategies, while
offline capabilities and the communication with hardware components are leading to problems. 
In the last chapter, the problem of hardware communication gets analyzed in detail
and a standardized convention based on the MQTT/Websocket protocol and the Homie convention 
is developed.\\
This thesis is addressed to software developers -- the reader should have a basic understanding of
software and web technologies. After reading this paper it should be clear how the development of
kiosk software with web technologies should be approached, which pros and cons they offer and
how a typical kiosk system can work with web technologies. 
